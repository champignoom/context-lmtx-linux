%&latex

\setupcolors[state=start]

\setupbodyfontenvironment[default][em=italic]

\definetype[typeTEX][option=color]
\setuptype[option=TEX]

%\setuphead[subsection][page=yes]

\define[1]\MyStyle{\color[blue]{\bf #1}}

\starttext

\section{Introduction}

The modules in the {\tt taspresent} directory aim to provide an
easy|-|to|-|use, consistent interface for writing simple presentations in
\CONTEXT. I had the idea to write this module when I was preparing my own
presentations with \CONTEXT. I wanted to be able to achieve visually
different results without changing my source files, so I wrote different
styles that followed the same logic and provided the same macros. When I
created this module, I had the following requirements in mind:

\startitemize
\item Most of the styles that are provided are rather sober in
  appearance. I use them for my university lectures in the humanities. They
  provide a nice but not too distracting background and a lot of space for
  presentations with large amounts of text.
\item The module is meant for presentations which will be shown with the
  help of a digital projector. Hence, they have no interactive elements
  (such as buttons) and no tools for navigation (such as a table of
  contents). 
\item The module allows for user configurability. It comes with several
  predefined styles and some predefined font options. The modular structure
  makes it easy to add further styles.
\item Picture placement and changing backgrounds is made easy by predefined
  macros.
\stopitemize

The module provides a simple, basic structure; I think it will be best suited
for beginners or intermediate users of \CONTEXT. It is definitely not meant
to compete with Hans's fuller and fancier presentation modules, and it
offers much less than the \LaTeX\ {\tt beamer} package. On the other hand,
it is much easier to use; you should be able to write your first
presentation after spending five minutes with this manual.

\section{Installation}

Installation is easy: just put the files {\tt t-<something>} into a
directory where \TeX\ can see them. For \CONTEXT\ third|-|party modules,
the canonical place would be in one of your {\tt TEXMF} trees, under {\tt
  tex/context/third}. If you want to keep things tidy, place them in a
subdirectory {\tt taspresent}. If you just unzip the archive {\tt
  taspresent.zip} in a {\tt TEXMF} directory, things will be moved to the
right place automatically. On many \TeX|-|systems, you will have to run
{\tt texhash} after installing new files. To doublecheck whether the system
finds your files, run {\tt kpsewhich t-taspresent.tex} from the command
line; if all goes well, this should return the position of the file you
have just installed.

\section{Setting up the Module}

To use the module, you put this line into your source file:

\startTEX
\usemodule[taspresent][style=,font=,size=,stylecolor=,stylebottom=]
\stopTEX

The values for the different keys will be explained in the following
sections. 

\section{The {\tt style} Key}

There are ten options for the {\tt style} key:

\subsection{\MyStyle{blackblue}}

This style was inspired by the \quotation{Copenhagen} theme of the \LaTeX\
{\tt beamer} package. The narrow blue and black stripes at the top and the
bottom of the slides display the date and slidenumber (top) and the title
and author of the presentation. 

\placefigure[here]{The \MyStyle{blackblue}
  style}{\externalfigure[blackblue][width=\textwidth]}

\page

\subsection{\MyStyle{bluegray}}

The colors of this style are very subdued and quiet; the interesting thing
is the pagenumber on the border of the margin and text area; this detail
was inspired by Hans's \quotation{split} style ({\tt pre-14}).

\placefigure[here]{The \MyStyle{bluegray}
  style}{\externalfigure[bluegray][width=\textwidth]}

\page

\subsection{\MyStyle{bluestripe}}

This theme is inspired by the \quotation{Berkeley} style of the \LaTeX\
{\tt beamer} package. Apart from the blue sidebars, it has no ornaments. 

\placefigure[here]{The \MyStyle{bluestripe}
  style}{\externalfigure[bluestripe][width=\textwidth]}

\page

\subsection{\MyStyle{darkshade}}

The only ornament to this style is the dark shaded background; it has two
style colors, {\tt stylecolor=blue} and {\tt stylecolor=green}; if you feel
really adventurous, try {\tt stylecolor=bluered}! It uses \CONTEXT's {\tt
  interactionbar} mechanism to show the progress of the presentation. It
provides much space for text.

\placefigure[here]{The \MyStyle{darkshade}
  style with {\tt stylecolor=blue}}{\externalfigure[blueshade][width=\textwidth]}

\page

\subsection{\MyStyle{doubleframe}}

This style was inspired by Hans's \quotation{green} style ({\tt s-pre-02}). It
has a thick blue frame around the entire slide area and a thinner frame around
the text area. The style has two options for the bottom area: {\tt
stylebottom=stripe} will display a shaded blue area which will grow with each
slide; {\tt stylebottom=square} displays a row of blue squares at the bottom
which also measure the presentation's progress.  

\placefigure[here]{The \MyStyle{doubleframe}
style with the {\tt stylebottom=stripe} option}{\externalfigure[blueframe][width=\textwidth]}

\page

\placefigure[here]{The \MyStyle{doubleframe}
style with the {\tt stylebottom=square} option}{\externalfigure[squareframe][width=\textwidth]}

\page

\subsection{\MyStyle{doubleshade}}

Similar to the \MyStyle{blueshade} style, but there is a differently shaded
area on the left with a progress meter. 

\placefigure[here]{The \MyStyle{doubleshade}
  style}{\externalfigure[doubleshade][width=\textwidth]}

\page

\subsection{\MyStyle{embossed}}

Spread the word, don't be shy! Show your pride in using \CONTEXT. The color
theme will probably look familiar; I copied it from the {\tt enattab}
manual. 

\placefigure[here]{The \MyStyle{embossed}
  style}{\externalfigure[embossed][width=\textwidth]}

\page

\subsection{\MyStyle{graybeams}}

This design is inspired by the {\tt husky} theme for the \LaTeX\ package {\tt
powerdot}, created by Jack Stalnaker.

\placefigure[here]{The \MyStyle{graybeams}
  style}{\externalfigure[graybeams][width=\textwidth]}

\page

\subsection{\MyStyle{graysquare}}

This minimalistic design is inspired by a presentation Taco gave at EuroTeX
2006.

\placefigure[here]{The \MyStyle{graysquare}
  style}{\externalfigure[graysquare][width=\textwidth]}

\page

\subsection{\MyStyle{greenblue}}

This style has cool colors and lots of white space; it is probably best suited
for presentations with relatively little text.

\placefigure[here]{The \MyStyle{greenblue}
  style}{\externalfigure[greenblue][width=\textwidth]}

\page

\subsection{\MyStyle{horizontalblue}}

A sober style with an emphasis on horizontal lines, inspired by {\tt
  beamer}'s \quotation{Szeged} theme. 

\placefigure[here]{The \MyStyle{horizontalblue}
  style}{\externalfigure[horizontalblue][width=\textwidth]}

\page

\subsection{\MyStyle{lightblue}}

This style emphasizes the title with its lively colors; the text area is
delimited by a light blue, shaded margin. 

\placefigure[here]{The \MyStyle{lightblue}
  style}{\externalfigure[lightblue][width=\textwidth]}

\page

\subsection{\MyStyle{narrowstripe}}

A variation on the \MyStyle{bluestripe} style, with shaded narrow
stripes. This style comes with two color options, which you set with the
{\tt stylecolor} key; {\tt stylecolor=red}, {\tt stylecolor=green}, or {\tt
  stylecolor=blue}.

\placefigure[here]{The \MyStyle{narrowblue}
style with {\tt stylecolor=blue}}{\externalfigure[narrowblue][width=\textwidth]}

% \page
% 
% \subsection{\MyStyle{narrowred}}
% 
% Like the \MyStyle{narrowblue} style, but with a red theme color.
% 
% \placefigure[here]{The \MyStyle{narrowred}
%   style}{\externalfigure[narrowred][width=\textwidth]}

\page

\subsection{\MyStyle{quadblue}}

This style is inspired by the colors and corporate look of my
university. It is very sober and offers much space for text and
images. There is a rough progress meter built into the blue quadrangles. 

\placefigure[here]{The \MyStyle{quadblue}
  style}{\externalfigure[quadblue][width=\textwidth]}

\page


\subsection{\MyStyle{rainbowstripe}}

A colorful style for daring presenters. The black line which marks the
progress is reminiscent of absorption lines in star spectra, so this style
may be apt for astrophysical presentations?

\placefigure[here]{The \MyStyle{rainbowstripe}
  style}{\externalfigure[rainbowstripes][width=\textwidth]}

\page

\subsection{\MyStyle{redframe}}

This style is inspired by the screen version of the Metafun manual. Watch
the small gray circles at the bottom!

\placefigure[here]{The \MyStyle{redframe}
  style}{\externalfigure[redframe][width=\textwidth]}

\subsection{Customization}

The style parameter allows easy customization. If you want to develop your
own theme, I would suggest copying one of the style files to another name,
say {\tt MyStyle.tex}, and modifying it to your heart's content: you can
change the colors or define a different background altogether, think of a
different way of displaying titles, different margins, etc. Just be sure to
define all the macros that are needed. After producing your own style,
again, copy it to a place where \TeX\ can find it (such as the same
directory where the source of your presentation resides) and instruct the
module to use your file:

\startTEX
\usemodule[taspresent][style=MyStyle,font=Times,size=17pt]
\stopTEX

\section{The {\tt font} Key}

There is a number of predefined fonts which can be selected by setting the
{\tt font} key. 

\starttabulate[|l|p|]
\NC {\tt LatinModern}     \NC typesets in LatinModern Serif\NC \NR
\NC {\tt LatinModernSans} \NC typesets in LatinModern Sans\NC \NR
\NC {\tt Times}           \NC the free clone of TimesNew Roman\NC \NR
\NC {\tt Helvetica}       \NC the free clone of Helvetica\NC \NR
\NC {\tt Pagella}         \NC the tex-gyre clone of Palatino; this will
only work if you have the tex-gyre fonts installed\NC \NR
\stoptabulate

In addition, there is a value {\tt User}; this will not set a font but
allow you to provide your own settings. If you set your own font, please
remember to select the bodyfont at \typeTEX{\Normalsize} and to give your setup
commands {\em after} loading the module (or \TeX\ will not know what
\typeTEX{\Normalsize} means and complain about an \quotation{undefined control
  sequence}). For example, for the samples included here, I have used my own
typescript which defines the Adobe MyriadPro font:

\startTEX
\usetypescriptfile[type-myriadpro]
\usetypescript[MyriadPro] [texnansi]
\setupbodyfont[MyMyriadPro,ss,\Normalsize]
\stopTEX

\section{The {\tt size} Key}

This selects the font size for the main text and defines a corresponding
size for titles.

\placefigure[here]{Text and title sizes}
{\setupTABLE[column][width=3cm,align=middle]
\bTABLE
\bTR \bTD Value \eTD \bTD Normalsize \eTD \bTD Titlesize \eTD \eTR
\bTR \bTD 16pt \eTD \bTD 16pt \eTD \bTD 25pt \eTD \eTR
\bTR \bTD 17pt \eTD \bTD 17pt \eTD \bTD 27pt \eTD \eTR
\bTR \bTD 18pt \eTD \bTD 18pt \eTD \bTD 28pt \eTD \eTR
\bTR \bTD 19pt \eTD \bTD 19pt \eTD \bTD 30pt \eTD \eTR
\bTR \bTD 20pt \eTD \bTD 20pt \eTD \bTD 30pt \eTD \eTR
\bTR \bTD 21pt \eTD \bTD 21pt \eTD \bTD 30pt \eTD \eTR
\eTABLE}

\section{Macros}

The module provides some macros to facilitate the preparation of
presentations. 

\subsection{\typeTEX{\setvariables}}

Begin your presentation by setting the name of the author(s) and the title
with this macro:

\startTEX
  
\setvariables [taspresent]
              [author={Groucho Marx},
               title={Marriage the Chief Cause of Divorce}]
\stopTEX

\subsection{\typeTEX{\Maketitle}}

This macro will produce a title page with the author and the title of the
presentation; the look is of course determined by the style of your
presentation. 

\placefigure[here]{A Title Page}{\externalfigure[titlepage][width=\textwidth]}

\subsection{\typeTEX{\Slidetitle{}}}

As the name suggests, this macro typesets its argument as the title of the
slide. What the title looks like is determined by the selected presentation
style. 

\subsection{\typeTEX{\PicHoriz}}

This macro facilitates the placement of landscape images. It takes two
arguments:

\startTEX
\PicHoriz[image][height=\NormalHeight]
\stopTEX

The first argument is the name of the image you want to place; the second
argument determines the size. If your picture is not too broad, a height of
\typeTEX{\NormalHeight} will make it fill up the entire text area. If your
picture is too broad, you should set \typeTEX{width=\textwidth}. 

\placefigure[here]{Placement of a horizontal picture}{\externalfigure[horizontal][width=\textwidth]}

\page

\subsection{\typeTEX{\PicVert}}

This macro facilitates the placement of portrait images. It takes three
arguments:

\startTEX
\PicVert[image][width=\NormalWidth]{Text \\ to go \\ with the picture}
\stopTEX

Again, the first argument is the name of the image you want to place; the
second argument determines the size. If your picture is not too high, a
width of \typeTEX{\NormalWidth} will make it fill up the entire left half
of the text area. If your picture is too hight, you should set
\typeTEX{height=\textheight}. The third argument is the text that will be
placed opposite the picture. 

\placefigure[here]{Placement of a vertical picture}{\externalfigure[vertical][width=\textwidth]}

\page

\subsection{\typeTEX{\CircHoriz}}

This command works exactly like \typeTEX{\PicHoriz}, but takes an
additional (third) argument. It places a red circle on top of the picture;
the placement and size of this circle is determined by this third argument: 

\startTEX
\CircHoriz[scale=40,x=120,y=80][image][height=\NormalHeight]
\stopTEX

The {\tt scale} key sets the diameter of the circle (in mm), {\tt x}
and {\tt y} set horizontal and vertical position. You will probably
have to fiddle with these keys to get the circle exactly where you want
it. 


\placefigure[here]{A picture with a red circle}{\externalfigure[circle][width=\textwidth]}

\page

\subsection{\typeTEX{\ArrowHoriz}}

This command works exactly like \typeTEX{\PicHoriz}, but takes an
additional (third) argument. It places a red arrow on top of the picture;
the direction and size of this arrow is determined by this third argument: 

\startTEX
\CircHoriz[direction=135,x=120,y=80][image][height=\NormalHeight]
\stopTEX

The {\tt direction} key sets the direction into which the arrowhead points,
{\tt x} and {\tt y} set its horizontal and vertical position. You will
probably have to fiddle with these keys to get the circle exactly where you
want it.

\placefigure[here]{A picture with a red arrow}{\externalfigure[arrow][width=\textwidth]}

\page

\subsection{\typeTEX{\CircVert} and \typeTEX{ArrowVert}}

Of course, there are also circles and arrows for \quotation{vertical}
pictures; again, the first argument is the position of the circle/arrow: 

\startTEX
\CircVert[scale=22,x=23,y=25]%
[vert]%
[width=\NormalWidth]%
{Circle in \\ Vertical Mode}

\ArrowVert[direction=90,x=7,y=23]%
[vert]%
[width=\NormalWidth]%
{Arrow in \\ Vertical Mode}
\stopTEX

\placefigure[here]{Vertical picture with red circle}{\externalfigure[circvert][width=\textwidth]}

\placefigure[here]{Vertical picture with red arrow}{\externalfigure[arrvert][width=\textwidth]}

\subsection{Background}

Some of the styles provide up to three backgrounds: for titles, for slides
with vertical image, and for normal slides with text or horizontal
images. Switching the backgrounds also adjusts parameters like margins or
headers, where this is necessary. There are three commands for setting the
background for title slides, \quotation{horizontal} slides and
\quotation{vertical} slides respectively:

\startTEX
\titback
\lecback
\picback
\stopTEX

\section{Handouts}

The easiest way to make handouts from your slides is post|-|processing the
pdf|-|file of your presentation. If you want to make the handouts from the
slides such as they appear, just run this command in the directory where
your slides are located (of course, put the name of your own presentation
where you see {\tt demo.pdf} in the example; the entire command has to go
in one long line):

{\tt texmfstart texexec --pdfcombine --combination='2*3' --nobanner

 --result=handout demo.pdf}

This way, you will get a handout with six slides typeset in two columns per
page. 

If you prefer to have the handouts without the colored background, typeset
your presentation and leave the {\em style} key unset; that way, you will
get a default version without backgrounds and fancy frames.

\stoptext

%%% Local Variables: 
%%% mode: context
%%% TeX-master: t
%%% End: 
