\mainlanguage[en]

% \enableregime[utf]

\usetypescript[palatino]
\setupbodyfont[palatino,11pt]

% \usetypescriptfile[type-palatino]
% \usetypescript[Palatino][texnansi]
% \setupbodyfont[MyPalatino,11pt]

% \setupinterlinespace[line=14pt]

% \usetypescriptfile[type-freeserif]

% \usetypescript[FreeSerif]

% \setupbodyfont[MyFreeSerif,11pt]

\usemodule[ancientgreek][font=Kadmos,scale=1.2,altfont=GreekGentiumAlt,scale=1.2]% GreekAsteria,scale=1.05,altfont=Kadmos]

\setupindenting[medium,yes]

\setupcolors[state=start]

\setupurl[style=normal]

\setupinteraction[state=start,color=blue,style=normal]

\usemodule[int-load]
\loadsetups[cont-en.xml]
\loadsetups[t-ancientgreek.xml]

\useURL[adobe][{http://www.adobe.com/products/reader/}][][Adobe]

\useURL[apa][{http://socrates.berkeley.edu/~pinax/greekkeys/index.html}][][GreekKeys
Page]

\useURL[arno][{http://www.adobe.com/type/browser/landing/arno/arno.html}][][Adobe]

\useURL[ellak][{http://www.ellak.gr/index.php?option=com_content\&task=view\&id=6493}][][ellak website]

\useURL[elpenor][{http://www.ellopos.net/elpenor/greek-texts/greek-fonts.asp}][][elpenor
website]

\useURL[freeserif][{http://savannah.nongnu.org/download/freefont/}][][savannah
website]

\useURL[garmprem][{http://www.adobe.com/type/browser/landing/garamond/garamond.html}][][Adobe]

\useURL[hartmann][{http://www.lucius-hartmann.ch/diverse/greekfonts/\#unicode}][][Lucius
  Hartmann's homepage]

\useURL[kerkis][{http://iris.math.aegean.gr/kerkis}][][University of the Aegean]

\useURL[kryukov][{http://www.thessalonica.org.ru/en/oldstandard.html}][][Alexey Kryukov]

\useURL[libertine][{http://linuxlibertine.sourceforge.net/Libertine-EN.html}][][sourceforge]

\useURL[monotype][{http://www.monotypefonts.com/Library/Non-Latin-Library.asp}][][Monotype
  Corporation]

\useURL[gentium][{http://scripts.sil.org/cms/scripts/page.php?site_id=nrsi&item_id=Gentium_download}][][gentium
site]

\useURL[gfs][{http://www.greekfontsociety.org/typefaces.html}][][Greek
  Font Society]

% \useURL[gdr][{http://www.ucl.ac.uk/GrandLat/greekfonts/frameFonts.html}][][University
%   College London]

% \useURL[gandh][{http://depts.washington.edu/ebmp/software.php}][][Early Buddhist Manuscript website]

% \useURL[labo][{http://www.geocities.com/SoHo/workshop/3799/download.htm}][][Carmelo Lupini's Laboratorio Pluri\-disciplinare]

% \useURL[myway][{http://dl.contextgarden.net/myway/GreekInContext.pdf}][][myway]

% \useURL[aisa][{http://www.oeaw.ac.at/kal/multikey}][][multikey website]

% \useURL[aristarcoj][{http://www.russellcottrell.com/greek/fonts.htm}][][Russell
%   Cottrell's site]

% \useURL[gainsford][{http://gainsford.tripod.com/archive/freeserif.zip}][][http://gainsford.tripod.com/archive/
%   freeserif.zip]

% \useURL[hancock][{http://www.users.dircon.co.uk/~hancock/vudown.htm}][][Ralph
% Hancock's site]

% \useURL[semata][{http://semata.delendis.com/}][][semata website]

% \useURL[cardo][{http://scholarsfonts.net/cardofnt.html}][][David J. Perry's
% website]

% \useURL[adobe][{http://www.adobe.com/products/acrobat/readstep2.html}][][adobe
% website]

% \useURL[rusten][{http://ccat.sas.upenn.edu/bmcr/1998/98.1.11.html}][][Jeffrey
% Rusten's review]

\useexternalfigure[theo][theo.jpg][width=.75\textwidth]

\definetype[typeTEX][option=tex,style=type]

\define[1]\ShowGreek%
{\bgroup\color[red]{\localgreek{#1}}\egroup}
\define[1]\ShowAltGreek%
{\bgroup\color[red]{\localaltgreek{#1}}\egroup}

% \enablemode[demo]

\starttext

\title{Greek in Proper \CONTEXT}

\section{For the Impatient}

\startitemize[n]

\item Unzip the {\tt zip} archive in one of your {\tt texmf} trees (the
  canonical places are {\tt texmf-local} or {\tt \$HOME/texmf}). 

\item In your sourcefile, use the module: \typeTEX+\usemodule[ancientgreek]+

\item Shorter Greek passages are included in the \typeTEX+\localgreek{}+ command,
  longer passages in the \typeTEX+\startgreek+ \textellipsis\ \typeTEX+\stopgreek+
  environment. 
  
\stopitemize

\section{Some Preliminary Remarks}

Until a few years ago, typesetting Greek with \TeX\ was a nontrivial
exercise because of the limitations of \TeX\ and of computer fonts. With
the advent of new font technologies (Truetype and OpenType fonts) and of
new \TeX\ engines (\XETEX\ and lua\TeX), this situation has changed. If you
are using these newer engines and are typesetting in a font with support
for polytonic Greek, your Greek text will appear without any further
steps. 

Hence, this Greek module is strictly necessary only if you want to typeset
Greek in \CONTEXT\ {\tt mkii}, i.e. good old pdf\TeX; for \XETEX\ and
lua\TeX, you can do without it. However, using the Greek module will offer
a number of advantages even with these new engines:

\startitemize[2]

\item You can change the font for Greek passages with one simple setup
  command.

\item You can easily mix a Latin font without support for Greek and a Greek
  font. 

\item The module allows you to scale the Greek font so that it provides a
  pleasant optical match with your Latin script.

\item The commands provided by the module also switch the language to
  \quotation{Greek} and provide proper hyphenation.

\item The module provides a few additional characters which will be handy
  especially for scholarly texts.

\stopitemize

At the time of this writing, \CONTEXT\ can use three \TEX\ engines:
traditional pdf\TeX, \XETEX, and lua\TeX. In theory, there should not be
any difference for the user; the same input file should give (nearly)
identical output, whether compiled with traditional {\tt texexec} command
(for pdf\TeX), the {\tt --xtx} switch (for \XETEX), or the {\tt context}
command (for lua\TeX). Utf-8 input is now considered the rule; older
solutions are merely treated in an appendix. As is already the case for
other parts of \CONTEXT, future development for the Greek module will be
restricted to the lua\TEX\ engine and {\tt mkiv}; support for the
traditional pdf\TEX\ engine ({\tt mkii}) will be frozen.

\section{Installation}

Installation of the module is straightforward: unpack the {\tt zip} archive
in one of your {\tt texmf}|-|trees.  If you have added these files to a
{\tt texmf}|-|tree that uses {\tt ls-R} files, you will need to rebuild
this database by running \type{texhash} (if you don't know what this means,
run it anyway, it won't hurt); if you're using mkiv, you need to run {\tt
  luatools --generate}. After this, everything should be in its proper
place.\footnote{For {\tt mkii} only: If your \CONTEXT|-|installation has
  \tex{autoloadmapfilestrue} set in \type{cont-sys.tex}, the module will
  take care of loading the appropriate mapfiles. If you have not enabled
  this setting or if you want to make the fonts available to other
  \TeX|-|applications, add the mapfile to the configuration of your
  postscript- and pdf|-|drivers: {\tt sudo updmap-sys --enable Map
    tasgreek.map}.}

\section{Usage}

Call the module in the preamble of your document and set up the necessary
fonts: 

\setup{ancientgreek}

The module lets you define a main Greek font with the {\tt font} parameter
which will be scaled at the {\tt scale} parameter and an alternative Greek
font with {\tt altfont} parameter which will be scaled at the {\tt
  altscale} parameter.

\subsection{The Fonts}

The module offers support for a variety of Greek fonts. For the casual
user, we can distinguish between three categories:

\page

\midaligned{Fonts that come with the module}

\starttabulate[|lw(.25\textwidth)|p|]

  \NC GreekCanonica \NC Free font from the \from[ellak]. \NC \NR

  \NC GreekLibertine \NC LinuxLibertine is a large OpenType font project
  whose main developer is Philipp H. Poll, hosted on \from[libertine]. \NC \NR

  \NC GreekOldStandard \NC OpenType font developed by \from[kryukov]; comes
  in upright and italic. \NC \NR

  \NC Ibycus \NC An OpenType version of the Ibycus font that was developed
  by Pierre McKay for the Ibycus|-|package for \LaTeX. No italics or
  bold. \NC \NR

\stoptabulate

\midaligned{Fonts that are freely available}

The following fonts are not included, either because of licensing issues or
because including them would have made the package too big. However, all
the support files are in place; all you need to do in order to use them is
download the appropriate files at the location indicated below, extract the
font files (in {\tt .ttf} or {\tt .otf} format), move them to the
appropriate directory in your installation of the Greek module ({\tt
  .../fonts/truetype/greek/...} or {\tt .../fonts/opentype/greek/...}), and
run {\tt texhash} (for {\tt mkii}) or {\tt luatools --generate} (for {\tt
  mkiv}). 

\starttabulate[|lw(.25\textwidth)|p|]

  \NC Alkaios \NC Can be downloaded from \from[hartmann] (there is no
  license information). \NC \NR

  \NC GreekAsteria \NC Can be downloaded at the \from[elpenor]. \NC \NR

  \NC GreekKerkis \NC Can be downloaded at the \from[kerkis]. \NC \NR

  \NC GreekMinion \NC The Minion font family comes with the free Adobe
  Reader software (downloadable at \from[adobe]). \NC \NR

  \NC FreeSerif \NC OpenType developed for Debian Linux by various
  contributors; can be downloaded from the \from[freeserif]. Complete
  font family. \NC \NR

  \NC GFS fonts \NC All the fonts beginning with \quotation{GFS} are
  relatively recent developments, available at the \from[gfs]. If you use a
  newer version of \TeX Live, many of them will already be present in your
  installation under {\tt .../texmf-dist/fonts/opentype/public/}; in this
  case, you should be able to just use them. If you're running the
  \CONTEXT\ minimals, you will have to download the font files. \NC \NR

  \NC Gentium \NC The Gentium and GentiumAlt fonts are included both in the
  minimals and in \TeX Live 2009, so if you use one of these distributions,
  they should just work. If you don't have them, they can be downloaded at
  the \from[gentium]. \NC \NR

\stoptabulate

\midaligned{Commercial Fonts}

The following fonts are commercial. The module offers all necessary support
for these fonts, but {\bf not the fonts themselves}. If you want to use
them, you will have to contact the license holders, buy an appropriate
license, and then copy the font files to your installation of the Greek
module. 

\starttabulate[|lw(.25\textwidth)|p|]

  \NC Bosporos \NC Comes with the GreekKeys keyboard macros (for Mac OS X
  and Windows), which can be purchased at the \from[apa]. \NC \NR

  \NC GreekArno \NC The ArnoPro font family, available at \from[arno]. \NC
  \NR 

  \NC GreekGaramondPrem \NC The Garamond Premier Pro font family, also
  available at \from[garmprem]. \NC \NR

  \NC GreekPorson \NC This is the original Porsonian font such as it has
  been used by many publishers (e.g., Oxford University Press) for a long
  time. It can be purchased from the \from[monotype]. \NC \NR

  \NC Kadmos \NC  Comes with the GreekKeys keyboard macros (for Mac OS X
  and Windows), which can be purchased at the \from[apa]. \NC \NR

  \NC NewHellenic \NC The original NewHellenic font such as it has been
  used by many publishers (e.g., Cambridge University Press) for a long
  time. It can be purchased from the \from[monotype]. \NC \NR

\stoptabulate

For the fonts Alkaios, Bosporos, GreekGentium, GreekGentiumAlt,
GreekPorson, Ibycus, Kadmos, and NewHellenic, an alternative
\quotation{lunate} (rounded) shape of the letter sigma
(\ShowGreek{\getglyph{name:GenAR102}{\char"0063}}, without distinction
between final and intermediate sigma) is available. Without making any
changes to your input, you can choose this form by adding \quotation{lun}
directly to the name of the font, e.g.,
\typeTEX+[font=GreekGentiumlun,scale=1.2]+.

Only a few fonts provide Roman, Italic, Bold, and BoldItalic variants; if
you need them, you will have to experiment whether the font switches
\typeTEX+\bf+, \typeTEX+\bi+ or \typeTEX+\em+ work. If the font doesn't
offer these variants, they are simply mapped to the normal
\quotation{Roman} form, so font switches will simply not give any visual
effect.\footnote{At a very early date, typesetters distinguished between
  an upright (\quotation{Roman}) and slanted (\quotation{Italic}) form of
  Latin letters; both forms were mixed for graphical effect and to add
  emphasis to parts of the text; whence our use of \quotation{Italics.}
  This convention did not exist for Greek typefaces; hence, Greek
  \quotation{italics} really is a spurious notion.}

\subsection{Scaling}

The module allows scaling of the Greek font. As you will probably know,
different fonts may look very different even at the same size: Lucida
Bright, e.g., at 10~pt is almost as big as Computer Modern at 12~pt. If you
want to mix fonts, you cannot simply rely on the design size; you have
adapt the different sizes by tweaking the appearance of the fonts. Since
our Greek fonts will be mixed with Roman text, the module provides the
ability to adapt the size of the Greek font: \type{[scale=1]} would
correspond to the design size of the font; \type{[scale=0.95]} would shrink
the font to 95\,\%; \type{[scale=1.05]} magnify to 105\,\%. Just play
around until the relation between both fonts looks right.

Since this scaling is done in relation to the main bodyfont of your
document, \CONTEXT\ needs a proper bodyfont environment to calculate
it. This environment is set up automatically for \quotation{normal} sizes
(9pt, 10pt, 11pt, 12pt). If you want to use a bodyfont at an unusual size
(say, at 19.5pt), you need to put these lines into the preamble of your
document {\bf before} defining the bodyfont:

\startTEX
\starttypescript [serif] [default] [size]
  \definebodyfont [19.5pt] [rm] [default]
\stoptypescript
\definebodyfontenvironment [19.5pt]
\stopTEX

Moreover, if you use an unusual bodyfont size and want Greek text in your
footnotes, you will need to set up their size explicitly (and add this size
to the \typeTEX+\definebodyfont+ typescript just described):

\startTEX
\setupfootnotes[bodyfont=14pt]
\stopTEX

\subsection{Greek Text}

After setting these values in the preamble, switching to Greek in the body
of your documents is easy: use the command \typeTEX+\localgreek{}+ for
shorter passages and the environment \typeTEX+\startgreek+ \textellipsis\
\typeTEX+\stopgreek+ for longer stretches (the corresponding commands for
the alternate Greek font are \typeTEX+\localaltgreek{}+ for shorter
passages and the environment \typeTEX+\startaltgreek+ \textellipsis\
\typeTEX+\stopaltgreek+ for longer stretches).

If you are using {\tt mkii}, you will see that certain environments
(esp. tables and tabulate) cannot be used within the \typeTEX+\startgreek
\stopgreek+ pair.\footnote{Technical explanation: in order for ASCII input
  to work, the Greek environment has to change some catcodes; this confuses
\TeX's table mechanisms.} In that case, you will have to wrap single table
cells into \typeTEX+\localgreek{}+ commands instead of having the entire
table in a  \typeTEX+\startgreek \stopgreek+ environment. This should not
affect {\tt mkiv} and \XETEX\ users.

Moreover, the module provides a couple of convenient commands to typeset
symbols that are needed for writing ancient Greek or editing papyrological
material:\footnote{Not all symbols are present in all fonts. In {\tt mkiv},
  there is a fallback mechanism which will try to take these symbols from
  fonts that do have them; this will work in most (but not all) cases. In
  {\tt mkii} and \XETEX , you will have to experiment with different
  fonts.}

\starttabulate[|lw(.4\textwidth)|lw(.1\textwidth)|lw(.4\textwidth)|lw(.1\textwidth)|]
\NC \typeTEX{\digamma} \NC \ShowGreek{\digamma} \NC \typeTEX{\sampi} \NC
\ShowGreek{\sampi}\NC \NR

\NC \typeTEX{\stigma} \NC \ShowGreek{\stigma} \NC \typeTEX{\koppa}  \NC
\ShowGreek{\koppa} \NC \NR 

\NC \typeTEX{\lunars}  \NC \ShowGreek{\lunars} \NC \typeTEX{\lunarS} \NC
\ShowGreek{\lunarS} \NC \NR

\NC \typeTEX{\textbraceleft} \NC \ShowGreek{\textbraceleft} \NC
\typeTEX{\textbraceright} \NC \ShowGreek{\textbraceright} \NC \NR

\NC \typeTEX{\halfbracketleft} \NC \ShowGreek{\halfbracketleft} \NC
\typeTEX{\halfbracketright} \NC \ShowGreek{\halfbracketright} \NC \NR

\NC \typeTEX{\doublebracketleft} \NC \ShowGreek{\doublebracketleft} \NC
\typeTEX{\doublebracketright} \NC \ShowGreek{\doublebracketright} \NC \NR

\NC \typeTEX{\anglebracketleft} \NC \ShowGreek{\anglebracketleft} \NC
\typeTEX{\anglebracketright} \NC \ShowGreek{\anglebracketright} \NC \NR

\NC \typeTEX{\crux} \NC \ShowGreek{\crux} \NC \NC \NC \NR
\stoptabulate

\section{Release History}

\setupindenting[no]

{\bf 04/2013: ver 1.3}

Split type-agr.tex into mkii and mkiv files.

\blank[small]

{\bf 02/2010: ver 1.2}

Fallback fonts for missing characters. Support for \XETEX\ and {\tt mkiv}
improved. 

\blank[small]

{\bf 08/2008: ver 1.1}

Support for \XeTeX\ and mkiv added. New mechanism for active characters. 

\blank[small]

{\bf 01/2007: ver 1.0}

New version of Aristarcoj font; added two new GFS fonts. The module has been
pretty stable for a while, so I decided to call it ver 1.0\textellipsis

\blank[small]

{\bf 11/2006: ver 0.99}

Added some new GFS fonts; removed fonts that have become unavailable.

General cleanup of fonts and encodings.

\blank[small]

{\bf 04/2006: ver 0.98}

Added the new GFSNeohellenic font.

Compatibility module {\tt t-oldgreek} now als uses the new {\tt
  moduleparameter} mechanism.

Cleaned up directory structure.

\blank[small]

{\bf 04/2006: ver 0.97}

Module now uses the new {\tt moduleparameter} mechanism. 

Renamed from {\tt t-greek} to {\tt t-ancientgreek} to avoid clashes in
namespace. 

Hyphenation of polytonic Greek now works properly.

\blank[small]

{\bf 02/2006: ver 0.95}

Supports more Greek fonts.

Greek are now defined as typefaces; normal font switches such as \type{\em}
and \type{\bf} work within Greek text.

\blank[small]

{\bf 05/2005: ver 0.9}

Initial release.

\section{Appendix: {\tt ASCII} input}

In the time before Unicode input, Greek was usually input via a
transliteration scheme developed for \LaTeX. This was available for
\CONTEXT\ mkii; it will not work with either mkiv or the \XETEX\
engine. Since this type of input is considered obsolete, I will just give a
very brief description of the correspondence between {\tt ASCII} input and
the typeset Greek letters:

\blank[big]

\setupTABLE[frame=off]

\setupTABLE[c][each][width=.0416\textwidth,align=middle]

\bTABLE

\bTR\bTD A\eTD\bTD B\eTD\bTD G\eTD\bTD D\eTD\bTD E\eTD\bTD Z\eTD\bTD H\eTD\bTD J\eTD\bTD I\eTD\bTD K\eTD\bTD L\eTD\bTD M\eTD\bTD N\eTD\bTD X\eTD\bTD O\eTD\bTD P\eTD\bTD R\eTD\bTD S\eTD\bTD T\eTD\bTD U\eTD\bTD F\eTD\bTD Q\eTD\bTD Y\eTD\bTD W
\eTD\eTR

\bTR\bTD\localgreek{Α}\eTD\bTD\localgreek{Β}\eTD\bTD\localgreek{Γ}\eTD\bTD\localgreek{Δ}\eTD\bTD\localgreek{Ε}\eTD\bTD\localgreek{Ζ}\eTD\bTD\localgreek{Η}\eTD\bTD\localgreek{Θ}\eTD\bTD\localgreek{Ι}\eTD\bTD\localgreek{Κ}\eTD\bTD\localgreek{Λ}\eTD\bTD\localgreek{Μ}\eTD\bTD\localgreek{Ν}\eTD\bTD\localgreek{Ξ}\eTD\bTD\localgreek{Ο}\eTD\bTD\localgreek{Π}\eTD\bTD\localgreek{Ρ}\eTD\bTD\localgreek{Σ}\eTD\bTD\localgreek{Τ}\eTD\bTD\localgreek{Υ}\eTD\bTD\localgreek{Φ}\eTD\bTD\localgreek{Χ}\eTD\bTD\localgreek{Ψ}\eTD\bTD\localgreek{Ω}\eTD\eTR

\bTR\bTD a\eTD\bTD b\eTD\bTD g\eTD\bTD d\eTD\bTD e\eTD\bTD z\eTD\bTD h\eTD\bTD j\eTD\bTD i\eTD\bTD k\eTD\bTD l\eTD\bTD m\eTD\bTD n\eTD\bTD x\eTD\bTD o\eTD\bTD p\eTD\bTD r\eTD\bTD s\eTD\bTD t\eTD\bTD u\eTD\bTD f\eTD\bTD q\eTD\bTD y\eTD\bTD w\eTD\eTR

\bTR\bTD\localgreek{α}\eTD\bTD\localgreek{β}\eTD\bTD\localgreek{γ}\eTD\bTD\localgreek{δ}\eTD\bTD\localgreek{ε}\eTD\bTD\localgreek{ζ}\eTD\bTD\localgreek{η}\eTD\bTD\localgreek{θ}\eTD\bTD\localgreek{ι}\eTD\bTD\localgreek{κ}\eTD\bTD\localgreek{λ}\eTD\bTD\localgreek{μ}\eTD\bTD\localgreek{ν}\eTD\bTD\localgreek{ξ}\eTD\bTD\localgreek{ο}\eTD\bTD\localgreek{π}\eTD\bTD\localgreek{ρ}\eTD\bTD\localgreek{σ}\eTD\bTD\localgreek{τ}\eTD\bTD\localgreek{υ}\eTD\bTD\localgreek{φ}\eTD\bTD\localgreek{χ}\eTD\bTD\localgreek{ψ}\eTD\bTD\localgreek{ω}\eTD\eTR

\eTABLE

Accents, breathings, and iota subscript are written with {\tt \quotesingle\
  \textgrave\ \textasciitilde\ < > \textbar}; these can be combined as
necessary. I provide just a few examples:

\blank[big]

{\tt >e} = \localgreek{ἐ}

{\tt \quotedbl \textasciitilde u} = \localgreek{ῧ}

{\tt <\textasciitilde w\textbar} = \localgreek{ᾧ}

\blank[big]

The original Greek encoding for \LaTeX\ took care of the letter sigma: when
it occurred at the end of a word, it would automatically be transformed
into a final sigma \localgreek{c}. After some hesitation, I decided not to
follow this approach: there are too many cases where users might want a
\quotation{normal} sigma even in front of a space or a punctuation mark. In
order to obtain a final sigma, type \quotation{c.}

\blank[big]

ASCII input was common and useful in {\tt mkii}; in {\tt mkiv}, which
expects Unicode input, it doesn't serve any purpose. Nevertheless, it can
be convenient if you want just a few words in Greek, to use this convention
(not all operating systems and all editors offer intuitive and ready
methods of inputting Unicode Greek). For this case (and this case alone),
the module offers a convenient shortcut: even in {\tt mkiv}, you can use
the command  \typeTEX+\asciigreek{}+. It will convert your ASCII input on
the fly into Unicode and process it with {\tt mkiv}. This is a hack and is
not meant for longer stretches of text. In particular, this conversion will
not be able to process \TeX\ commands within the  \typeTEX+\asciigreek{}+
group, so use at your own risk. 

\stoptext

Since some of these symbols (especially the papyrological brackets) are
very rare in Greek fonts, I have decided to use glyphs from Latin Modern
and mathematical fonts that should be available on every \TeX\ system.

In order to get an apostrophe, type {\tt \quotesingle \quotesingle}. For
opening and closing quotes, type {\tt ((} and {\tt ))}. To obtain the
sublinear dot in papyrological or epigraphical editions, type an
exclamation mark after the character: {\tt >a!n!\quotesingle h!r!} becomes
\localgreek{>a!n!'h!r!}.

Unicode input is intuitive: if you have a proper keyboard driver for input
of Unicode Greek, just type away, but remember to include Greek passages
between \type{\localgreek {...}} or \type{\startgreek \stopgreek}. Of
course, all the named glyphs are available for Unicode|-|input as well.

Please be advised that not all fonts contain all symbols or
characters. Archaic number symbols, the sublinear dot, or characters that
do not occur in normal Attic morphology (like \localgreek{>~e}) are not
included in some of the fonts, so if you need those, you will have to
experiment which font provides them (Ibycus, KadmosNew and GreekCardo are
pretty complete).

\stoptext
