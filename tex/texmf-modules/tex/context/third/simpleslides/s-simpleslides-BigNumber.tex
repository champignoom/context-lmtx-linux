%D \module
%D   [      file=simpleslides-s-BigNumber,
%D        version=2022.07.13,
%D          title=\CONTEXT\ Style File,
%D       subtitle=Presentation Module --- Big Number Style,
%D         author=Aditya Mahajan and Thomas A. Schmitz,
%D           date=\currentdate
%D      copyright={Aditya Mahajan and Thomas A. Schmitz}]
%C
%C Copyright 2007 Aditya Mahajan and Thomas A. Schmitz
%C This file may be distributed under the GNU General Public License v. 2.0.

%D This file provides the \quotation{BigNumber} style for the presentation
%D module. It is loaded at runtime.

\writestatus{simpleslides}{loading Big Number style}

\startmodule[simpleslides-s-BigNumber]

\unprotect

%D First, we change the page layout to have more space all around

\setuplayout [width=fit,
              margin=2cm,
              height=fit,
	      leftmargindistance=.8cm,
	      rightmargindistance=0cm,
              header=18mm,
              footer=0cm,
              topspace=.8cm,
              backspace=1.9cm,
              location=singlesided]

\setuplayout [simpleslides:layout:horizontal][header=18mm]
\setuplayout [simpleslides:layout:vertical]  [header=0mm]
\setuplayout [simpleslides:layout:title]     [header=0mm]

\setuplayer[simpleslides:layer:slidetitle]
    [width=\paperwidth,
    height=\paperheight,x=20mm]

%D Next we define generic frames, which will be used by other macros to
%D get a consistent look and feel.

\defineframed[simpleslides:framed:small]
             [frame=off,offset=0pt,
              width=1.7cm,align=middle]

\setupcombinations[distance=2.5em]

%D These macros are used for placing figures/pictures:

\define\NormalHeight        {\textheight}
\define\NormalWidth         {.46\textwidth}
\define\PictureFrameHeight  {\textheight}
\define\PictureFrameWidth   {.46\textwidth}

%D This module has two color schemes, a blue one and a red one.

\startsetups simpleslides:setups:blue
\definecolor [simpleslides:contrastcolor]      [r=0.8,g=0.8,b=0.9]
\definecolor [simpleslides:backgroundcolor]	   [s=.88]
\definecolor [simpleslides:textcolor]          [s=0]
\stopsetups

\startsetups simpleslides:setups:red
\definecolor [simpleslides:contrastcolor]     [r=0.4]
\definecolor [simpleslides:backgroundcolor]	  [s=.35]
\definecolor [simpleslides:textcolor]         [s=1]
\stopsetups

%D Now we choose the scheme that the user asked for

\doifsetupselse{simpleslides:setups:\moduleparameter{simpleslides}{color}}
    {\setups{simpleslides:setups:\moduleparameter{simpleslides}{color}}}
    {\setups{simpleslides:setups:blue}}

\setupcolors[textcolor={simpleslides:textcolor}]


%D The characteristic feature of this module is that the page number is drawn in
%D big letters on the slide. First we define the font used to draw the number.

\definefontsynonym  [BigNumberFont] [name:texgyreherosbold]

\definefont         [NumberFont]    [BigNumberFont at 30pt]


%D We use \METAPOST\ to draw backgrounds. First, we define a few helper macros
%D to place text inside \METAPOST

\definetextext[simpleslides:sometxt:left] {\TaspresentSometxtLeft}
\definetextext[simpleslides:sometxt:right]{\TaspresentSometxtRight}

\unexpanded\def\TaspresentSometxtLeft#1%
  {\getvalue{simpleslides:framed:small}
                       {\color[simpleslides:contrastcolor]
                       {\NumberFont #1}}}

\unexpanded\def\TaspresentSometxtRight#1%
  {\getvalue{simpleslides:framed:small}
                       {\NumberFont \color[simpleslides:backgroundcolor]{#1}}}

%D Now we define a \METAPOST| graphic that draws the number. The exact
%D dimensions have been found by trial and error.

\startuseMPgraphic{simpleslides:MP:ornament}
StartPage ;
save Left, Right ;
picture Left, Right ;

Left  := textext("\NumberFont \color[simpleslides:contrastcolor]{\pagenumber}") ysized 4cm ;
Right := textext("\NumberFont \color[simpleslides:backgroundcolor]{\pagenumber}") ysized 4cm ;

save LeftBox, RightBox ;
path LeftBox, RightBox ;

save split ; numeric split ;
split := if RealPageNumber < 10 : 1/2 else : 3/4 fi  ;
%split := 1/2 ;

LeftBox := llcorner Left -- split[llcorner Left, lrcorner Left]
           -- split[ulcorner Left, urcorner Left] -- ulcorner Left --cycle ;

RightBox := lrcorner Right -- split[lrcorner Right, llcorner Right]
           -- split[urcorner Right, ulcorner Right] -- urcorner Right --cycle ;

save shft ; numeric shft ;
shft = arclength(llcorner Right -- lrcorner Right) ;
if RealPageNumber >= 10 :
  RightBox := RightBox shifted  (shft/2, 0) ;
fi;

clip Left  to LeftBox  ;
clip Right to RightBox ;

save corner; pair corner ;
corner := lrcorner Field[Text][Text] shifted (.2cm,0.3cm) ; % same as x2

labeloffset := 0bp;

label.ulft(Left,  corner) ;
label.urt (Right, corner) ;

StopPage ;
\stopuseMPgraphic

%D We also use \METAPOST\ to draw the horizontal and vertical page backgrounds.

\startuniqueMPgraphic{simpleslides:MP:horizontal}
StartPage ;
fill Page withcolor \MPcolor{simpleslides:contrastcolor} ;
fill Field[Text][Text] enlarged 0.2cm
     withcolor \MPcolor{simpleslides:backgroundcolor} ;
StopPage ;
\stopuniqueMPgraphic

\startuniqueMPgraphic{simpleslides:MP:vertical}
StartPage ;
fill Page withcolor \MPcolor{simpleslides:contrastcolor} ;

z1 = urcorner Field[Text][Text] shifted (.2cm,0) ;
z2 = lrcorner Field[Text][Text] shifted (.2cm,-.2cm) ;
z3 = z1 shifted (-8.05cm,0) ;
z4 = (x3,y2) ;

save Main ;
path Main ;
Main := z1 -- z2 -- z4 -- z3 --cycle ;

fill Main withcolor \MPcolor{simpleslides:backgroundcolor} ;
StopPage ;
\stopuniqueMPgraphic


%D We define these backgrounds as overlays:

\defineoverlay
  [simpleslides:background:horizontal]
  [\useMPgraphic{simpleslides:MP:horizontal}]

\defineoverlay
  [simpleslides:background:vertical]
  [\useMPgraphic{simpleslides:MP:vertical}]

\defineoverlay
  [simpleslides:background:title]
  [\useMPgraphic{simpleslides:MP:horizontal}]

\defineoverlay
  [simpleslides:background:ornament]
  [\useMPgraphic{simpleslides:MP:ornament}]

%D The slide title is placed on the top of the text area. The layer takes care
%D of the positioning.

\setupSlideTitle
   [\c!after=,
    \c!alternative=layer,
    \c!width=\textwidth,
    \c!height=2.5cm,
    \c!command=\doSlideTitle]

\setupTitle
   [\c!headcolor=simpleslides:textcolor,
   \c!title\c!color=simpleslides:textcolor,
   \c!date\c!color=simpleslides:textcolor,
   \c!author\c!color=simpleslides:textcolor]

%D Squares are used as the first level of itemizations

\definesymbol[1][$\square$]
\setupitemize[1][inmargin]
%\setupitemize[each][joinedup,unpacked]

\protect
\stopmodule

\endinput
