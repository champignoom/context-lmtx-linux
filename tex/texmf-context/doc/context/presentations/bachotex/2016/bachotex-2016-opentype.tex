\usemodule[present-banner]

\startdocument
  [title={OPENTYPE FONTS},
   subtitle={the generic loader},
   location={Hans Hagen \endash\ bacho\TeX\ 2016}]

\starttitle[title={how engines sees a font}]

\startsubject[title={\TeX}]

    \highlight [nb] {fields:} width, height, depth, italic correction, kern table,
    ligature tree, vf commands, next size pointer, extensible specification
    \highlight [nb] {and} a set of text and math parameters

\stopsubject

\startsubject[title={\pdfTeX}]

    \highlight [nb] {extra fields:} left protruding, right protruding, expansion
    factor \highlight [nb] {and} parameters to control these

\stopsubject

\startsubject[title={\LuaTeX}]

    \highlight [nb] {extra fields:} math top accent, math bot accent, tounicode,
    adapted extensible specification, vertical variants, horizontal variants,
    name, index, used status, math kerns \highlight [nb] {and} extra parameters
    \highlight [nb] {and} math constants \highlight [nb] {and} no 8~bit
    limitations

\stopsubject

\startsubject[title={\XeTeX}]

    probably something similar

\stopsubject

\stoptitle

\starttitle[title={font handling}]

\startsubject[title={loading opentype font data}]

    \startitemize
        \startitem
            till recently we used the built|-|in fontforge loader library
        \stopitem
        \startitem
            but now we use a recently written \Lua\ loader
        \stopitem
        \startitem
            but use a similar feature handler
        \stopitem
        \startitem
            in \ConTeXt\ one can fall back to the old loader/handler
        \stopitem
    \stopitemize

\stopsubject

\startsubject[title={applying (opentype) features}]

    \highlight [nb] {generic modes:} base, node \crlf
    \highlight [nb] {\ConTeXt\ modes:} base, node, auto, dynamic

\stopsubject

\startsubject[title={locating (opentype) fonts}]

    \startitemize
        \startitem
            \highlight [nb] {file}: kpse in generic, resolvers in \ConTeXt
        \stopitem
        \startitem
            \highlight [nb] {name}: simple in generic, extended in \ConTeXt,
            different in \LaTeX
        \stopitem
        \startitem
            \highlight [nb] {spec}: not in generic (uses font database)
        \stopitem
        \startitem
            \highlight [nb] {virtual}: not in generic
        \stopitem
        \startitem
            \highlight [nb] {lua}: delegated to low level interfaces
        \stopitem
    \stopitemize

\stopsubject

\stoptitle

\starttitle[title={preparations}]

\startsubject[title={after loading}]

    \startitemize
        \startitem
            initialize format driven substitution
        \stopitem
        \startitem
            initialize format driven positioning
        \stopitem
        \startitem
            enable analysis of states/properties
        \stopitem
        \startitem
            initialize additional data for engine (protrusion, expansion, extend,
            slant)
        \stopitem
        \startitem
            apply user or \TeX\ format extensions
        \stopitem
        \startitem
            apply manipulations before and after loading
        \stopitem
        \startitem
            (build virtual fonts)
        \stopitem
        \startitem
            enable special script handlers (fuzzy side of opentype)
        \stopitem
        \startitem
            pass metrics and some metadata to \TeX
        \stopitem
    \stopitemize

\stopsubject

\startsubject[title={benefit}]

    efficient access to all font properties for additional processing beforehand
    or afterwards

\stopsubject

\stoptitle

\starttitle[title={processing}]

\startsubject[title={steps}]

    \startitemize
        \startitem
            (comes after hyphenation)
        \stopitem
        \startitem
            first identifies to be handled modes
        \stopitem
        \startitem
            normalization (in \ConTeXt) node list
        \stopitem
        \startitem
            delegate handling to \TeX\ or \Lua
        \stopitem
        \startitem
            when using \Lua\ features are applied in prescribed order:
            substitution, positioning, etc.
        \stopitem
        \startitem
            as last step positioning is finalized (left/right kern injection,
            space kerning, anchoring, cursives)
        \stopitem
    \stopitemize

\stopsubject

\startsubject[title={remarks}]

    \startitemize
        \startitem
            efficient contextual analysis is|-|non trivial
        \stopitem
        \startitem
            discretionaries need special care: ...pre ...replace... post...
        \stopitem
        \startitem
            there is no real limit in extensions
        \stopitem
        \startitem
            it's not too hard to inject experimental code
        \stopitem
        \startitem
            so users can add their own features
        \stopitem
        \startitem
            some day there may be alternative handlers
        \stopitem
    \stopitemize

\stopsubject

\stoptitle

\starttitle[title={math}]

\startsubject[title={format}]

    the opentype math specification stays close to \TeX, but has extensions and
    more control (see articles & presentations by Ulrik Vieth)

\stopsubject

\startsubject[title={loading}]

    \startitemize
        \startitem
            maps more or less directly onto internal structures
        \stopitem
        \startitem
            in \ConTeXt\ we use(d) virtual unicode fonts awaiting lm/gyre
        \stopitem
    \stopitemize

\stopsubject

\startsubject[title={processing}]

    character mapping and special element handling remains macro package
    dependent

\stopsubject

\startsubject[title={construction}]

    \startitemize
        \startitem
            we split code paths when needed: traditional or opentype (no longer
            heuristics)
        \stopitem
        \startitem
            the \luaTeX\ engine provides much control over spacing and a bit more
            over rendering
        \stopitem
    \stopitemize

\stopsubject

\stoptitle

\starttitle[title={the basics of loading}]

\startsubject[title={the format}]

    \startitemize
        \startitem
            it evolved out of competing formats by apple, microsoft and adobe
        \stopitem
        \startitem
            two flavours can normally be recognized by suffix: \type {ttf} and
            \type {otf}
        \stopitem
        \startitem
            main differences are bounding box info, global kern tables, cubic vs
            quadratic curves
        \stopitem
        \startitem
            multiple sub fonts inside \type {ttc} files (font collections)
        \stopitem
        \startitem
            it's considered a standard (so it should be possible to implement)
        \stopitem
    \stopitemize

\stopsubject

\startsubject[title={the specification}]

    \startitemize
        \startitem
            the only useable reference is on the microsoft website
        \stopitem
        \startitem
            (the iso mpeg standard is more or less a bunch of ugly rendered
            webpages)
        \stopitem
        \startitem
            trial and error helps understanding/identifying fuzzy aspects
        \stopitem
    \stopitemize

\stopsubject

\stoptitle

\starttitle[title={the available loaders}]

\startsubject[title={the fontforge loader}]

    \startitemize
        \startitem
            offers the same view on the font as the editor (good for debugging)
        \stopitem
        \startitem
            in order to process a font some optimal data structures are created
            after loading
        \stopitem
        \startitem
            we cache fonts because loading and creating these structures takes
            time and it saves memory too
        \stopitem
        \startitem
            fontforge has a lot of heuristics (catching issues collected over
            time) but these are hard to get rid of when they're wrong
        \stopitem
    \stopitemize

\stopsubject

\startsubject[title={the lua loader}]

    \startitemize
        \startitem
            this started out as experiment for loading outlines in \MetaFun
        \stopitem
        \startitem
            it avoids the conversion to optimal structures for handling
        \stopitem
        \startitem
            we can hook in better heuristics (data is more raw)
        \stopitem
        \startitem
            it fits in the wish for maximum flexibility (next stage \ConTeXt)
        \stopitem
        \startitem
            it's rather trivial to extend and adapt without hard coding
        \stopitem
        \startitem
            the performance can be a bit less on initial loading (pre|-|cache)
            but there is a bit of room to improve
        \stopitem
        \startitem
            it's much more efficient in identifying fonts (not a real issue in
            practice)
        \stopitem
        \startitem
            in practice most fonts behave ok (no recovery needed) but there are
            some sloppy fonts around
        \stopitem
    \stopitemize

\stopsubject

\stoptitle

\starttitle[title={what do we load}]

\startsubject[title={tables}]

    \startitemize
        \startitem
            opentype is mostly tables with lots of subtables
        \stopitem
        \startitem
            there are required, truetype outline, postscript outline, (svg and
            bitmap), typography & additional ones
        \stopitem
        \startitem
            the typographic tables specify transformations to apply (gdef, gsub,
            gpos)
        \stopitem
    \stopitemize

\stopsubject

\startsubject[title={calculations}]

    \startitemize
        \startitem
            as we need ht/dp we need to calculate the boundingbox of postscript
            outlines (cff parser)
        \stopitem
        \startitem
            internally we use unicodes instead of indices
        \stopitem
        \startitem
            we need to identify/filter the right unicode information
        \stopitem
        \startitem
            we want to do more so we need to carry around more info (tounicode etc)
        \stopitem
    \stopitemize

\stopsubject

\startsubject[title={pitfalls}]

    \startitemize
        \startitem
            there is no real consistent approach to use of basic features:
            single, one to multiple, multiple to one & many to many replacements,
            and look ahead and/or back based solutions
        \stopitem
        \startitem
            in principle consistent families like lm/gyre could share common data
            and logic but otherwise there is much diversity around
        \stopitem
    \stopitemize

\stopsubject

\stoptitle

\starttitle[title={a few details}]

\startsubject[title={loading}]

    \startitemize
        \startitem
            load the file (subfont if needed) in a \Lua\ friendly format
        \stopitem
        \startitem
            prepare for later processing and/or access
        \stopitem
        \startitem
            optimize data structures
        \stopitem
        \startitem
            cache the instance (and compile to bytecode)
        \stopitem
        \startitem
            share loaded font data where possible
        \stopitem
        \startitem
            initialize & mark enabled features
        \stopitem
        \startitem
            pass metrics, parameters and some properties to \TeX
        \stopitem
    \stopitemize

\stopsubject

\startsubject[title={processing}]

    \startitemize
        \startitem
            we need to run over enabled features (also virtual non|-|opentype
            ones)
        \stopitem
        \startitem
            we use lookup hashes to determine if action is needed
        \stopitem
        \startitem
            if needed we access detailed data and apply it
        \stopitem
        \startitem
            there can be a few but also many hundreds of loops over the node list
        \stopitem
        \startitem
            contextual matching can make us end up with a real lot of access and
            analysis
        \stopitem
        \startitem
            descending into discretionaries adds significant overhead (so it's
            optimized)
        \stopitem
    \stopitemize

\stopsubject

\stoptitle

\starttitle[title={traditional fonts}]

\startsubject[title={tfm}]

    \startitemize
        \startitem
            there is a built|-|in loader for \type {tfm}, \type {ofm}, \type {vf}
            and \type {ovf} files
        \stopitem
        \startitem
            encoding and filename mapping is as usual (\type {enc} and \type
            {map} files)
        \stopitem
        \startitem
            (in the early days \ConTeXt\ filtered info from those \type {enc}
            files too)
        \stopitem
    \stopitemize

\stopsubject

\startsubject[title={type one}]

    \startitemize
        \startitem
            type one fonts have their own loader that gets information from \type
            {afm} files
        \stopitem
        \startitem
            the \type {pfb} file is consulted to get the index (to unicode)
            mapping
        \stopitem
        \startitem
            the \type {afm} loader was already written in \Lua\ but we now can also use
            \Lua\ for the \type {pfb} file
        \stopitem
    \stopitemize

\stopsubject

\stoptitle

\starttitle[title={remarks}]

    \startitemize
        \startitem
            features like additional character kerning don't belong in the font
            handler as they are (to some extent) macro package dependant
        \stopitem
        \startitem
            the same is true for italic correction (often input related and
            therefore a macro package specific issue)
        \stopitem
        \startitem
            setting up protrusion and expansion is again somewhat macro package
            dependent
        \stopitem
        \startitem
            \ConTeXt\ has many extra font related mechanisms and features
            (described in a more technical manual)
        \stopitem
        \blank
        \startitem
            this has to work well with the core subsystems: languages especially
            hyphenators, specific script demands, typesetting (all kind), builders
            (paragraph, page), etc.
        \stopitem
        \startitem
            a complication is that we do this more and more in \Lua, but still need to
            support the built|-|in mechanismsm too
        \stopitem
        \blank
        \startitem
            the interfacing to macro packages differs (for plain \TeX\ we use
            code that ships with \ConTeXt)
        \stopitem
        \startitem
            for bugs and issues of with fonts in \ConTeXt\ you use its mailing list (or
            mail me)
        \stopitem
        \startitem
            the \LaTeX\ interface is handled by Philipp Gesang
    \stopitemize

\stoptitle

\starttitle[title={future}]

    \startitemize
        \startitem
            we'll improve handling of border cases (within the constraints of
            performance)
        \stopitem
        \startitem
            we might provide a few more hooks for plug|-|ins
        \stopitem
        \startitem
            the type one \type {pfb} reader will be extended to provide outlines
            (not complex, needed for \MetaFun)
        \stopitem
        \startitem
            we keep playing with extra new features and virtual fonts
        \stopitem
        \blank
        \startitem
            maybe some more code can be made generic (fwiw)
        \stopitem
    \stopitemize

\stoptitle

\starttitle[title={credits}]

    \startitemize
        \startitem
            Kai Eigner and Ivo Geradts for (experimental) patches in the handlers
            for rare, complex & creepy fonts
        \stopitem
        \startitem
            Philipp Gesang for binding the generic code to \LaTeX\ font mechanims.
        \stopitem
        \startitem
            Idris Samawi Hamid for testing and providing the very complex and
            demanding Husayni font
        \stopitem
        \startitem
            Hartmut Henkel for the initial cleaning up of expansion and protrusion
        \stopitem
        \startitem
            Taco Hoekwater for the original loader and discussions and a lot more
        \stopitem
        \startitem
            Boguslaw Jackowski and friends for the fonts and patience with us
        \stopitem
        \startitem
            Dohyun Kim for testing and suggestions on CJK font support
        \stopitem
        \startitem
            Mojca Miklavec for distributions, managing us, and basically everything
        \stopitem
        \startitem
            Luigi Scarso for patiently testing and managing my patches and testing
            very beta code
        \stopitem
        \startitem
            Thomas Schmitz for using betas in deadline critital book production
            and making sure we patch fast
        \stopitem
        \startitem
            Ton Otten for permitting me to work on all this \TeX\ related stuff for
            ever and ever (and using to the extreme)
        \stopitem
        \startitem
            Wolfgang Schuster for knowing and testing every detail of \ConTeXt\
            and writing selectfont (for system fonts)
        \stopitem
        \blank
        \startitem
            and all (\ConTeXt) users who patiently accept betas and testing
        \stopitem
    \stopitemize

\stoptitle

\stopdocument
