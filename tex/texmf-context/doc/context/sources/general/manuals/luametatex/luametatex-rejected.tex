% language=us runpath=texruns:manuals/luametatex

\environment luametatex-style

\startcomponent luametatex-rejected

\startchapter[reference=rejected,title={Rejected features}]

\startsection[title=Introduction]

I should have started this chapter sooner because some experiments were removed
without them being documented. This chapter is mostly for myself so that I don't
revisit rejected features.

\stopsection

\startsection[title=Text]

{\em todo}

\stopsection

\startsection[title=Math]

\startsubsection[title=Fences]

The \tex {mathfencesmode} primitive could be used to force the atom class of the
fake fences to be the old inner class instead of the new fence class but we
dropped that: it's now always a fence subtype (class). Unpacking is therefore now
controlled by a class option and not enforced by the (new in \LUAMETATEX) subtype.

\stopsubsection

\startsubsection[title=Delimiters]

The \tex {mathdelimitersmode} primitive was experimental and dealt with the
following (potential) problems. Three bits can be set. The first bit prevents an
unwanted shift when the fence symbol is not scaled (a cambria side effect). The
second bit forces italic correction between a preceding character ordinal and the
fenced subformula, while the third bit turns that subformula into an ordinary so
that the same spacing applies as with unfenced variants.

So, when set to 7 fenced subformulas with unscaled delimiters came out the same
as unfenced ones. This could be handy for cases where one was forced to use \prm
{left} and \prm {right} always because of unpredictable content. The full list of
flags that we had at the time of removal is given in the next table:

\starttabulate[|c|l|]
\DB value  \BC meaning \NC \NR
\TB
\NC \type{"01} \NC don't apply the usual shift \NC \NR
\NC \type{"02} \NC apply italic correction when possible \NC \NR
\NC \type{"04} \NC force an ordinary subformula \NC \NR
\NC \type{"08} \NC no shift when a base character \NC \NR
\NC \type{"10} \NC only shift when an extensible \NC \NR
\NC \type{"20} \NC don't error on a missing right  \NC \NR
\LL
\stoptabulate

Because the effect depends on the font (and for Cambria one could use for
instance \type {"16}) we finally decided to kick it out and correct the font
instead using the font goodie file. After all it's more a \CONTEXT\ preference
than an overall demand (read: we think no one else cares much about this).

The \type {"20} options to not error on a missing right fence has been turned
into \prm {mathcheckfencesmode}.

\stopsubsection

\startsubsection[title=Flattening]

The \tex {mathflattenmode} primitive is gone as we have other ways to deal with
this now. It related to italic corrections in traditional fonts.

\stopsubsection

\startsubsection[title=Rules]

The \tex {mathrulethicknessmode} feature has been turned into a class option
which is more granular.

\stopsection

\startsubsection[title=Control]

Although it has its use when developing \LUAMETATEX\ the \tex {mathcontrolmode}
parameters is no longer there. We have plenty of (more) detailed control now.

\stopsection


\stopchapter

\stopcomponent
