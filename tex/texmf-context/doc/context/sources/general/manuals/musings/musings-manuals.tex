% language=us runpath=texruns:manuals/musings

% \showfontkerns

% \setupalign[granular]

\startcomponent musings-manuals

\environment musings-style

\startchapter[title={About manuals}]

\startsection[title={Introduction}]

I'm always puzzled when I read that someone wonders if \CONTEXT\ is still up to
date or maintained because some manual has a timestamp of a decade ago. I'm also
puzzled by some rants you can run into when searching the web. In the next few
paragraphs I'll comment on this.

\stopsection

\startsection[title={Stability}]

Say that you're an enthusiastic user of console commands like \type {ls} (\type
{dir}), \type {cp} (\type {copy}) or maybe \type {ssh}, \type {rsync}, \type
{curl}. How often do you consult a manual on how they evolve? And say that you,
for some reason, do consult a manual, there is a good change that it is
pretty old. Does that mean that the commands are obsolete? The binaries probably
get fixed for bugs but the interface stays the same, which is what you expect.
Every time we generate a zip for the \CONTEXT\ distribution, the related website
also gets generated, using a bunch of \XML\ files that get transformed to \HTML\
using \XSLT\ and a pretty ancient version of \type {xsltproc} (why should I
update). I never check for a new manual as it keeps doing the job. And additional
manuals and reports get added.

So, once some functionality is stable, and a lot of macro code in \CONTEXT\ is
just that, there is no need to update a manual! Putting a new time stamp on it is
basically fake updating. And often the more introductionary kind of manuals don't
need to be updated at all, apart from maybe cultural changes that demand a
(political correct) update. Them being a bit old and not being updated is
actually a good thing as it signals stability.

It is worth mentioning that the \CONTEXT\ distribution is not the only source of
information. There are manuals written by others and there is the Wiki. All is
the work of volunteers and updating all that depends on how much time one can
allocate.

\stopsection

\startsection[title={Excuses}]

It is a fact that \CONTEXT\ evolves. New functionality gets added and some
mechanism get extended. Often these are described in dedicated manuals or
articles that end up in collections, and there are plenty of them in the
distribution. For some reason those complaining about a beginners manual with an
old time stamp don't check if there is more, and there is quite some more! Don't
only look at the \CONTEXT\ garden (the wiki) but also keep an eye on what gets
distributed. Some users are very good in track of what gets added, because
sometimes I get fixes for typos send within a few hours after uploading a zip.

We appreciate that other users point out that writing manuals takes time and that
indeed our time is not without limits. If I could sit down and write manuals
whole day, and it would get paid, I might do it. But it is a fact that
development of \CONTEXT\ is not paid for at all. I can work on it in company time
but much happens in spare time. Most development is a gamble on future use or
done because we want to be complete or because code can be improved. So, writing
a manual then closely relates to what we like doing: it determines the topics and
priorities. If something gets explored and ends up in new functionality then that
gets documented in the process. It is the fun factor that drives it. The same is
true for \LUATEX\ development.

So, we have as valid excuse that new manuals relate to (new) functionality and
old ones stay as they are. Don Knuth remarks somewhere that writing a manual as
part of the development is a good thing. We fully agree with that.

\stopsection

\startsection[title={Cutting edge}]

Does an old manual indicate that nothing happens? Definitely not. Over a decade
of \LUATEX\ development is closely related to \CONTEXT\ and there is plenty of
reporting about that. Does that mean that we need to rewrite manuals? No,
existing functionality remains. And of course users are free to come up with
more detailed manuals (which they seldom do). Some developments get published
in user group journals but we don't publish much about specific \CONTEXT\ features
and usage because it's hard to do that for a diverse audience.

Currently we have what is called \CONTEXT\ \MKXL\ (aka \LMTX), but we also have
the prelude to that, \MKIV, and the frozen predecessor \MKII. Apart from changes in
technology (most noticeably fonts and encodings) the functionality is accumulative:
most old manuals (unless they are specialized into old school fonts for instance)
apply to the latest greatest version.

It is a misunderstanding that the development of \CONTEXT, \LUATEX\ and
\LUAMETATEX\ is somehow funded by projects that we do. This is not true. We can
apply both in projects but as we charge by the hour (or day) no customer ever
sees development on the bill. Of course during a project we can gain on
efficiency (so then development pays back) and because we know the system style
writing is efficient too. In fact, in most cases our customers don't know or care
what tool we use because tools are expected to be part of the deal. Most projects
we can (and could) only do because we can use \CONTEXT\ and that is a side effect
of the fact that we do develop beforehand. We're often the only technically
and|/|or affordable way out. It's a chicken|-|egg issue: we have a tool and
therefore get a project. We never get a projects where we can develop a tool. No
one pays for \TEX\ development or at least no one ever came to us with specific
\TEX\ related demands. It looks like the world takes it for granted that \TEX\ is
just there. \footnote {There are a few subcomponents of \CONTEXT\ that were
partially sponsored by users and we do have some support contracts that permit
experiments and development.}

The reason for \CONTEXT\ being cutting edge (in terms of \TEX) is that we like
challenges, that users demand features that are interesting to explore and that
we've been part of the \TEX\ scenery for a while now. We just like that.

It's good to know that \CONTEXT\ was and is developed as a toolkit. We started
long ago because we needed a way to quickly create and update reports of meetings
that we chaired. Next we needed a way to efficiently produce high quality
education materials (of various kind) and support maintenance of sets of related
(quality assurance) manuals. We could have used wordstar, wordperfect or msword
but liked the \TEX\ way much more. As said, most customers didn't even know or
care what tool was used because the (often highly interactive \PDF) outcome
mattered most. In fact, we would not be interested in this kind of work if we
were forced to use clumsy tools, but for sure a lot can be done with those as
well.

\stopsection

\startsection[title={Continuity}]

Most development happens at \PRAGMA\ by me (Hans) with help from my two
colleagues (Ton and Kees) and the community (Aditya, Alan, Mikael, Mojca, Luigi,
Hraban, Taco, Thomas, Tomas, Willi, Wolfgang, and others). \footnote {More names
could be here as I write this in 2022.} I won't mention those on the mailing list
who contribute with ideas, testing and support, but they can't be missed. The
biggest danger for continuity is a polluted code base where everyone just pushed
code into a repository. So this is closely guarded. A user patch might work well
for that user but can break something else.

With \TEX\ you need to keep in mind that once a solution works there is no need
to update code or manuals. As long as there is a working \LUATEX\ (\LUAMETATEX)
binary you're fine. Maybe if some specific fonts are used, a filename might need
to be adapted.

An example. When we added a new \XML\ subsystem to \CONTEXT\ \MKIV\ we knew that
some day we could use it. We now uses it in a few projects and I'm pretty sure
that we would not do these projects otherwise as it would demand writing quite
complex \XSLT\ style sheets that then would have to be applied to thousands of
files per run. To some extend what is available in \CONTEXT\ sort of drives the
kind of work you look for. That said: if you consider using \CONTEXT\ for simple
or complex documents, either of not in a collection, either or not using \TEX\ or
\XML\ input you can be assured that this will work (and might even get better)
because we use it ourselves.

If you want to get an idea about development, just look at the (five) documents
that describe the development of \LUATEX\ (\LUAMETATEX). Locating them in the
distribution is a good opportunity to explore the documents. They will show you
what happened the last decade(s) and give you some trust in \CONTEXT. Or come to
a \CONTEXT\ meeting and meet those involved.

\stopsection

% \startsection[title={Stay away}]
%
% There are reasons why you won't find us (me) on some social media. First of all
% we like using the mailing list. It puts some time between what one thinks or
% wants to ask and actually getting it out in the open. After all, one might regret
% a too spontaneous complain or even rant, so it sort of protects subscribers.
%
% It is not that hard to search the web (say Reddit) and find someone complaining
% about something \TEX\ (or \CONTEXT) related, including \type {f**ck} kind of
% wording. If you're in that category, please stay away from \CONTEXT\ or try to
% constrain yourself. It won't trigger help. Why on earth does one rant about a
% system that one doesn't like unless maybe one is enforced to use it. We prefer 10
% out of 10000 \TEX\ users using \CONTEXT\ over 1000 of which 990 are unhappy: use
% another macro package and be part of the possible unhappy crowd there. We're not
% selling a product, we're sharing a potential useful tool! Also keep in mind that
% (as I know a bit about \TEX\ and its ecosystem) it would be very easy for me to
% counter you if I really wanted that; it might make you look ridiculous.
%
% That said, we see no benefit in being on Facebook. I never look for something
% there myself I'm too unfamiliar with it, I have nothing to share and and don't
% look forward to keeping something up to date. There is no need for an instant
% news channel on Whatsapp, Instagram, Telegram, Tiktok or other media that are
% driven by advertisements and filtering user data. We could post tutorials on
% YouTube but don't. It is already bad enough that I have to endure advertisements
% myself, let alone that users get bombarded by unrelated crap. We could have a
% fancy website and try to sell \CONTEXT\ \quote {professional} or present some
% \quote {plans}, but we like to keep it simple: the site is generated when we push
% an update and has been for decades so little work is involved in that. We don't
% keep statistics, let alone have trackers. I see no benefit in it and I don't
% like to waste time on that either.
%
% We are not on Stack Exchange but other \CONTEXT\ developers post very nice
% \CONTEXT\ answers there. Personally I just don't like a system with likes, votes,
% counters and somewhat persistent old (sometimes confusing) stuff. But I'm
% thankful for others who can deal with it.
%
% We could have a Patreon account but that's not how we work here and from seeing
% how it is used it would make me feel uncomfortable anyway. In a similar fashion
% we don't see benefits in commercializing the Github repository. Those who need
% support or and want to support development can find us if they want to.
%
% We have always participated in \TEX\ meetings and publish in use group journals
% about developments and such. We also have always been pretty active in the
% community and like it there. Just stay away from it when ranting is all you can
% contribute.
%
% \stopsection

\startsection[title={Closing remark}]

So next time someone asks if \CONTEXT\ is maintained because some old manual
stays around, return the question if frequently updated manuals are a sign of
stability. Also ask if someone looked a bit in the documentation tree. The oldest
manual in the \TEX\ world is the \TEX book that describes the oldest stable set
of macros: plain \TEX. There are happy users out there who love that stability.
If it were not for the wonderful personality of Don Knuth this program would
already been forgotten. I think that long term stability and unchanged code and
manuals are something that we need to cherish and get accustomed to, which is not
easy in a time when a phone and its operating system are outdated as soon as you
unbox it. It's also not easy in a time of instant communication, more and more
confused by what is called artificial intelligent mumbling, but that's for
another wrapup.

\stopsection

\startlines
Hans Hagen
Hasselt NL

(uncorrected so there's something left to complain)
\stoplines

\stopchapter

\stopcomponent
