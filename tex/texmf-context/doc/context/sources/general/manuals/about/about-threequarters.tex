% language=us

\startcomponent about-calls

\environment about-environment

\logo[CRITED]{CritEd}

\startchapter[title={\LUATEX\ 0.79}]

% Hans Hagen, PRAGMA ADE, April 2014

\startsection[title=Introduction]

To some it might look as if not much has been done in \LUATEX\ development but
this is not true. First of all, the 2013 versions (0.75-0.77) are quite stable
and can be used for production so there is not much buzz about new things.
\CONTEXT\ users normally won't even notice changes because most is encapsulated
in functionality that itself won't change. The binaries on the \type
{contextgarden.net} are always the latest so an update results in binaries that
are in sync with the \LUA\ and \TEX\ code. Okay, behaviour might become better
but that could also be the side effect of better coding. Of course some more
fundamental changes can result in temporary bugs but those are normally easy to
solve.

Here I will only mention the most important work done. I'll leave out the
technical details as they can be found in the manual and in articles that were
written during development. The version discussed is 0.79.

\stopsection

\startsection[title=Speed]

One of the things we spent a lot of time on is speed. This is of course of more
importance for a system like \CONTEXT\ that can spend more than half its time in
\LUA, but eventually we all benefit from it. For the average user it doesn't
matter much if a run takes a few seconds but in automated workflows these
accumulate and if a process has to produce 5 documents of 20 pages (each
demanding a few runs) or a few documents of several hundreds of pages, it might
make a difference. In the \CRITED\ project we aim for complex documents produced
from \XML\ at a rate of 20 pages per second, at least for stock \LUATEX.
\footnote {This might look slow but a lot is happening there. A simple 100 page
document with one word per page processes at more that 500 pages per second but
this is hard to match with more realistic documents. When processing data from
bases using the \CLD\ interface getting 50 pages per seconds is no problem.} In
an edit|-|preview cycle it feels better if we don't use more than half a second
for a couple of pages: loading the \TEX\ format, initializing the \LUA\ modules,
loading fonts, typesetting and producing a proper \PDF\ file. We also want to be
prepared for the ultra portable computers where multiple cores compensate the
lower frequency, which harms \TEX\ as sequential processor using one core only.

An important aspect of speedup is that it must not obscure the code. This is why
the easiest way to achieve it is to use a faster variant of \LUA, and \LUAJIT\
with its faster virtual machine, is a solution for that. We are aware of the
fact that processors not necessarily become faster, but that on the other hand
memory becomes larger. Disk speed also got better with the arrival of
flash based storage. Because \LUATEX\ should run smoothly on future portable
devices, the more we can gain now, the better it gets in the future. A decent
basic performance is possible and we don't have to focus too much on memory and
disk access and mostly need to keep an eye on basic \CPU\ cycles. Although we
have some ideas about improving performance, tests demonstrate that \LUATEX\
is not doing that bad and we don't have to change it's internals. In fact, if we
do it might as well result in a drastic slowdown!

One interesting performance factor is console output. Because \TEX\ outputs
immediately with hardly any buffering, it depends a lot on the speed of console
output. This itself depends on what console is used. \UNIX\ consoles normally
have some buffering and refresh delay built in. There the speed depends on what
fonts are used and to what extend the output gets interpreted (escape sequences
are an example). I've run into cases where a run took seconds more because of a
bad choice of fonts. On \WINDOWS\ it's more complicated since there the standard
console (like \TEX) is unbuffered. The good news is that there are several
alternatives that perform quite well, like console2 and conemu. These
alternatives buffer output and have refresh delays. But still, on a very high res
screen, with a large console window logging has impact. Interesting is that when
I run from the editor (SciTE) output is pretty fast, so normally I never notice
much of a slowdown. Of course these kind of performance issues can hit you more
when you work in a remote terminal.

The reason why I mention this is that in order to provide a user feedback about
issues, there has to be some logging and depending on the kind of use, more or
less is needed. This means that on the \CONTEXT\ mailing list we sometimes get
complaints about the amount of logging. It is for this reason that much logging is
optional and all logging can be disabled as well. Because we go through \LUA\
we have some control over efficiency too. In the current \LUATEX\ release most
logging can now be intercepted, including error messages.

Talking of a slowdown, in the \CRITED\ project we have to deal with real large
indices (tens of thousands of entries) and we found out that in the case of
interactive variants (register entry to text and back) the use of \LUAJITTEX\
could bring down a run to a grinding halt. In the end, after much testing we
figured out that a suboptimal string hashing function was the culprit and we did
extensive tests with both the \LUAJIT, \LUA\ 5.1 and \LUA\ 5.2 variant. We ended
up by replacing the \LUAJIT\ hash function by the the \LUA\ 5.1 one which is a
relative easy operation. Because \LUAJIT\ can address less memory than regular
\LUA\ it will always be a matter of testing if \LUAJITTEX\ can be used instead of
\LUATEX. Standard document processing (reports and such) is normally no problem
but processing large amounts of data from databases can be an issue.

In the process of cleaning up the code base for sure we will also find ways to
make things run even smoother. But, in any case, version 0.80 is already a good
benchmark for what can be achieved.

\stopsection

\startsection[title=Nodes]

One of the bottlenecks in the hybrid approach is crossing the so called C
boundary. This is not really a bottleneck, unless we're talking of many millions
of function calls. In practice this only happens in for instance more extreme
font handling (Devanagari or sometimes Arabic). If performance is really an issue
one can fallback on a more direct node access model. Of course the overhead of
access should be compared to other related activities: one can gain .25 seconds
on a run in using the direct access model, but if the whole runs takes 25
seconds, it can be neglected. If the price paid for it is less readable code it
should only be done deep down a macro package where no user even sees the code.
We use this access model in the \CONTEXT\ core modules and so far it looks quite
okay, especially for more extensive manipulations. The gain in speed is quite
noticeable if you use the more advanced features of \CONTEXT.

There can be some changes in the node model but not that drastic as the current
model is quite ok and also stays close to original \TEX\ so that existing
documentation still applies. One of the changes will be that glue spec (sub)nodes
will disappear and glue nodes will carry that information. Direction whatsits
will become first class nodes as they are part of the concept (whatsits
normally relate to extensions) and the same might happen with image nodes. As a
side effect we can restructure the code so that it becomes more readable. Some
experimental \PDFTEX\ functionality will be removed as it can be done better with
callbacks.

\stopsection

\startsection[title=The parbuilder and HZ]

As we started from \PDFTEX\ we inherit also its experimental code and character.
One of the objectives is to separate font- and backend as good as possible. We
have already achieved a lot and apart from bringing consistency in the code, the
biggest change has been a partial rewrite of the hz code, especially the way
fonts are managed. Instead of making copies of fonts with different properties,
we now carry information in the relevant nodes. The backend code already got away
from multiple fonts by using transformation of the base font instead of
additional font instances, so this was a natural adaptation. This was actually
triggered by the fact that a \LUA\ based par builder demonstrated that this made
sense. The new approach uses less memory and is a bit faster (at least in
theory).

In callbacks it makes life easier when a node list has a predictable structure.
For instance, the result of a paragraph broken into lines still has discretionary
nodes. Is that really needed? Lines can have left- or rightskip nodes, depending
on the fact if they were set. Math nodes can disappear as part of a cleanup in
the line break code, but this is unfortunate when one expects them to be
somewhere in the list in a callback. All this will be made consistent. These are
issues we will look into on the way to version 1.0.

I occasionally play with the \LUA\ based par builder and it is quite compatible
even if we take the floating point \LUA\ aspect into account. However when using
hz the outcome is different: sometimes better, sometimes worse. Personally I
don't care too much as long as it's consistent. Features like hz are for special
cases anyway and can never be stable over years if only because fonts evolve. And
we're talking of bordercase typesetting: narrow columns that no matter what method is
used will never look okay. \footnote {Some people don't like larger spaces, others
don't like stretched glyphs.}

\stopsection

\startsection[title=The backend]

The separation of front- and backend is more a pet project. There is some
experimental code that will get removed or integrated. We try to make the backend
consistent from the \TEX\ as well as \LUA\ end and some is reflected in
additional features and callbacks.

Some of the variables that can be set (the \LUA\ counterparts of the \type {\pdf..}
token registers at the \TEX\ end) are now consistent with each other and avoid
going via pseudo tokenization.  Typical aspects of a backend that only a few users
will notice but nevertheless needed work.

The merge of engines also resulted in inconsistencies in function names, like using
\type {pdf_} in function names where nothing \type {PDF} is involved.

\stopsection

\startsection[title=Backlinks]

In callbacks we mostly deal with node lists. At the \TEX\ end of course we also
have these lists but there it is quite clear what gets done with them. This means
that there is no need for double linked lists. It also means that what is known
as the head of a list can in fact be in the middle. The for \TEX\ characteristic
nesting model has resulted in stacks and current pointers. The code uses so
called temp nodes to point at the head node.

As a consequence in \LUATEX, where we present a double linked list, before the
current version one could run into cases where for instance a head node had a
prev pointer, even one that made no sense. As said, no big deal in \TEX\ but in
the hands of a user who manipulates the node list it can be dramatic. The current
version has cleaned head nodes as well as consistent backlinks, but of course we
keep the internals mostly unchanged because we stay close to the Knuthian
original when possible. \footnote {Even with extensions the original
documentation still covers most of what happens.}

\stopsection

\startsection[title=Properties]

Sometimes you want to associate additional information to a node. A natural way
to do this is attributes. These can be set at the \TEX\ and \LUA\ end and
accessed at the \LUA\ end. At the \LUA\ end one can have tables with nodes as
indices and store extra information but that has the disadvantage that one has no
clue if such information is current: nodes come and go and are recycled.

For this reason we now have a global properties table where each allocated node
can have a table with whatever information users might like to store. This itself
is not special, but the nice thing is that when a node is freed, that information
is also freed. So, you cannot run into old data. When nodes are copied its
properties are also copied. The overhead, when not used, is close to zero, which is
always an objective when extending the core engine.

Of course this model demands that macro package somehow controls consistent use
but that is not different from what already has to be done. Also, simple
extensions like this avoid hard codes solutions, which is also something we want
to avoid.

\stopsection

\startsection[title=\LUA\ calls]

We have so called user nodes that can carry a number, string, token list or node
list. We now have added \LUA\ to this repertoire. In fact, we now could use only a
\LUA\ variable and we might have done so in retrospect, but for the moment we we
stick to the current model of several basic types. The \LUA\ variable can be
anything and it is up to the user (in some callback) to deal with them.

User nodes are not to be confused with late \LUA\ nodes. You can store a function
call in a user node but that's about it. You can at a later moment decide to call
that function but it's still an explicit action. The value of a late \LUA\ node
on the other hand is dealt with automatically during shipout. When the value is a
string it gets interpreted as \LUA, but new is that when the value is a function
it will get called. At that moment we have access to some of the current backend
properties, like locations.

\stopsection

\startsection[title=Artefacts]

Because \LUATEX\ took code from \PDFTEX, that is built upon \ETEX, which in turn
is an extension to \TEX, and \OMEGA, that also extends \TEX, there is code that
no longer makes sense for us. Combine that with the fact that some code carries
signatures of translated \PASCAL\ to \CCODE, we have some cleanup to do as follow
up on the not to be underestimated move to \CCODE. This is an ongoing process but
also fun doing. Luigi and I spend many hours exploring venues and have
interesting Skype sessions that can easily sidetrack, and with Taco getting more
time for \LUATEX\ we expect to get most of our (still growing) todo list done.

Because \LUATEX\ started out as an experiment, there is some old code around. For
instance, we used to have multiple instances and this still shows in some places.
We can simplify the \LUA\ to \TEX\ interface a bit and clean up the \LUA\ global
state handling, but we're not in a big hurry with this. Experiments have been
done with some extensions to the writer code but they are hold back to after the
cleanup.

In a similar fashion we have sped up the way \LUA\ keyword and values get
resolved. Already early in the development we did this for critical code like
passing \LUA\ font tables to \TEX, followed by accessing nodes, but now we have
done that for most code. There is still some to do but it has the side effect of
not only consistency but also of helping to document the interface. Of course we
learn a lot about the \LUA\ internals too. The C macro system is of great help
here, although the mentioned pascal conversion (web2c) and merged engines have
resulted in some inconsistency that needs to be cleaned up before we start
documenting more of the internals (another subproject we want to finish before
retirement).

\stopsection

\startsection[title=Callbacks]

There are a few more callbacks and most of them come from the tracker. The
backend now has page related callbacks, the \LUA\ error handler can be
intercepted. Error messages that consist of multiple pieces are handled better
too. When a file is opened and closed a callback is now possible. Technically we
could have combined this with the already present callbacks but as in \TEX\
synchronization matters these new callbacks relate to current message callbacks
that show \type {[]}, \type {{}}, \type {<>} and|/|or \type {<<>>} fenced
filenames, where the later were introduced in successive backend code.

\stopsection

\startsection[title=\LUA]

We currently use \LUA\ 5.2 but a next version will show up soon. Because \LUA\
5.3 introduces a hybrid number model, this will be one of the next things to play
with. It could work out well, because \TEX\ is internally integer based (scaled
points) but you never know. It could be that we need to check existing code for
serialization and printing issues but normally that will not lead to
compatibility issues. We could even decide to stick to \LUA\ 5.2 or at least wait
till all has stabilized. There is some basic support for \UTF\ in 5.3 but in
\CONTEXT\ we don't depend on that. In practice hardly any processing takes place
that assumes that \UTF\ is more than a sequence of bytes and \LUA\ can handle
bytes quite well.

\stopsection

\startsection[title=\CONTEXT]

Of course the development of \LUATEX\ has consequences for \CONTEXT. For
instance, existing code is used to test alternative solutions and sometimes these
make it into the core. Some new features are used immediately, like the more
consistent control over \PDF\ properties, but others have to wait till the new
binary is more widespread. \footnote {Normally dissemination is rather fast
because the contextgarden provides recent binaries. The new windows binaries
often show up within hours after the repository has been updated.}

Some of the improvement in the code base directly relate to \CONTEXT\ activities.
For instance the \CRITED\ project (complex critical editions) uncovered some
hashing issues with \LUAJIT\ that have been taken care of now. The (small)
additions to the \PDF\ backend resulted in a partial cleanup of relatively old
\CONTEXT\ backend code.

Although some more complex mechanisms, like multi|-|columns are being reworked,
it is still needed to open up a bit more of the \TEX\ internals, so we have some
work to do. As usual, version 0.80 doesn't mean that only 0.20 has to be done to
get to 1.00, as development is not a linear process. The jump from 0.77 to 0.79
for instance involved a lot of work (exploration as well as testing). But as long
as it's fun to do, time doesn't matter much. As we've said before: we're in no
hurry.

\stopsection

\stopchapter

\stopcomponent
