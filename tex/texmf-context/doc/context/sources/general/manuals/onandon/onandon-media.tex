% language=us

\startcomponent onandon-media

\environment onandon-environment

\startchapter[title={The state of \PDF}]

\startsection[title={Introduction}]

Below I will spend some words on the state of \PDF\ in \CONTEXT\ mid 2018. These
are just some reflections, not an in|-|depth discussion of the state of affairs. I
sometimes feel the need to wrap up.

\stopsection

\startsection[title={Media}]

For over two decades \CONTEXT\ has supported fancy \PDF\ features like movies and
sound. In fact, as happens more, the flexibility of \TEX\ made it possible to
support such features right after they became available, often even before other
applications supported them.

The first approach to support such media clips was relatively easy. In \PDF\ one
has the text flow, resulting from the typesetting process, either or not enhanced
with images that are referred to from the flow. In that respect images are an
integral part of \PDF. On a separate layer there can be annotations. There are
many kinds and they are originally a sort of extension mechanism that permits
plugins to add features to a document. Examples of this are hyperlinks and the
already mentioned media clips. Video was supported by the quicktime movie plugin.
As far as I know in the meantime that plugin has been dropped as official part of
Acrobat but one can still plug it in.

Later an extra mechanism was introduced, tagged renditions. It separates the
views from the media and was more complex. When I first played with it, quite
some media were possible, and I made a demo that could handle mov, mp3, smi and
swf files. But last time I checked none of these really worked, apart from the
swf file. One gets pop|-|ups for missing viewers and a look at the reader
preferences makes one pessimistic about future support anyway. But one should be
able to set up a list of useable players with this mechanism (although only an
Adobe one seems to be okay so we're back to where we started).

At some point support for u3d was added. Interesting is that there is quite some
infrastructure described in the \PDF\ standard. Also something called rich media
was introduced and that should replace the former video and audio annotations
(definitely in \PDF\ version 2) and probably some day the renditions will no
longer be supported either. Open source \PDF\ viewers just stuck to supporting
text and static images.

Now, do these rich media work well? Hardly. The standard leaves it to the viewer
and provides ways to define viewers (although it's unclear to me how that works
out in practice.) Basically in \PDF\ version 2 there is no native support for
simple straightforward video. One has to construct a complex set of related
annotations.

One can give arguments (like security risks) for not supporting all these fancy
features but then why make rich media part of the specification at all? Browsers
beat \PDF\ viewers in showing media and as browsers can operate in kiosk mode I
suppose that it's not that hard to delegate showing whatever you want in an
embedded window in the \PDF\ viewer. Or why not simply support videolan out of
the box. All we need is the ability to view movies and control them (play, pause,
stop, rewind, etc). Where \HTML\ evolved towards easier media support, \PDF\
evolved to more obscurity.

So, how bad is it really? There are \PDF\ files around that have video! Indeed,
but the way they're supposed to do this is as follows: currently one actually has
to embed a shockwave video player (a user interface around something built|-|in)
and let that player show for instance an mp4 movie. However, support for
shockwave (flash) will be dropped in 2020 and that renders documents that use it
obsolete. This even makes one wonder about \JAVASCRIPT\ and widgets like form
fields, also a rather moving and somewhat unstable target. (I must have a
document being a calculator somewhere made in the previous century, in the early
days of \PDF.)

I think that the plugin model failed already rather early in the \PDF\ history if
only because it made no sense to develop them when in a next version of Acrobat
the functionality was copied in the core. In a similar fashion \JAVASCRIPT\
support seems to have stalled.

Unfortunately the open source viewers never catched on with media, forms and
\JAVASCRIPT\ and therefore there has been no momentum created to keep things
supported. It all makes efforts spent on supporting this kind of \PDF\ features a
waste of time. It also makes one careful in using them: it only works on the
short term.

Get me right, I'm not talking of complex media like 3d or animations but of
straightforward video support. I understand that the rich media framework tries
to cover complex cases but it's simple cases that carry the format. On the other
hand, one can wonder why the \PDF\ format makes it possible to specify behaviour
that in practice depends on \JAVASCRIPT\ and therefore could as well have been
delegated to \JAVASCRIPT\ as well. It would probably have been much cleaner.
\footnote {It looks like mu\PDF\ in 2018 got some support related to widgets aka
fields but alas not for layers which would be quite useful.}

The \PDF\ version 2 specification mentions \type {3D}, \type {Video} and \type
{Audio} as primary content types so maybe future viewers will support video out
of the box. Who knows. We try to keep up in \CONTEXT\ because it's often not that
complex to support \PDF\ features but with hardly any possibility to test them,
they have a low priority. And with Acrobat moving to the cloud and thereby
creating a more of less lifelong dependency on remote resources it doesn't become
much interesting to explore those routes either.

\stopsection

\startsection[title={Accessibility}]

A popular \PDF\ related topic is accessibility. One aspect of that is tagged
\PDF. This substandard is in my opinion not something that deserves a price for
beauty. I know that there are \CONTEXT\ users who need to be compliant but I
always wonder what a publisher really does with such a file. It's a bit like
requiring \XML\ as source but at the same time sacrificing really rich encoded
and sources for tweaks that suite the current limitations of for instance browsers,
tool|-|chains and competence. We've seen it happen.

Support for tagged \PDF\ has been available in \CONTEXT\ already for a while but
as far as I know only Acrobat professional can do something with it. The reason
for tagging is that a document is then useable for (for instance) visually
impaired users, but aren't they better served with a proper complete and very
structured source in some format that tools suitable for it can use? How many
publishers distribute \PDF\ files while they can still make money on prints? How
many are really interested in distributing enriched content that then can be
reused somehow? And how many are willing to invest in tools instead of waiting
for it to happen for free? It's a bit cheap trick to just expect authors (and
their in the case of \TEX\ free tools) to suit a publishers needs. Anyway, just
as with advanced interactive documents or forms, I wonder if it will catch on. At
least no publisher ever asked us and by the time they might do the competition of
web based dissemination could have driven \PDF\ to the background. But, in
\CONTEXT\ we will keep supporting such features anyway, if only because it's
quite doable. But \unknown\ it's user demand that drives development, not the
market, which means that the motivation for implementing such features depends on
user input as well as challenging aspects that make it somewhat fun to spend time
on them.

\stopsection

\startsection[title={Quality assurance}]

Another aspect popping up occasionally is validation. I'm not entirely sure what
drives that but delegating a problem can be one reason. Often we see publishers
and printers use old versions of \PDF\ related tools. Also, some workflows are
kind of ancient anyway and are more driven by \POSTSCRIPT\ history than \PDF\
possibilities. I sometimes get the impression that it takes at least a decade for
these things to catch on, and by that time it doesn't matter any more that \TEX\
and friends were at the front: their users are harassed by what the market
demands by then.

Support for several standards related to validation is already part of \CONTEXT\
for quite a while. For instance the bump from \PDF\ 1.7 to 2.0 was hardly worth
noticing, simply because there are not that many fundamental changes. Adapting
\LUATEX\ was trivial (and actually not really needed), and macro packages can
provide what is needed without much problems. So, yes, we can support it without
much hassle. Personally I never ran into a case where validation was really
needed. The danger of validation is that it can give a false impression of
quality. And as with everything quality control created a market. As with other
features it is users who drive the availability of support for this. After all,
they are the ones testing it and figuring out the often fuzzy specifications.
These are things that one can always look at in retrospect (like: it has to be
done this or that way) while in practice in order to be an early adopter one has
to gamble a bit and see where it fails or succeeds. Fortunately it's relatively
easy to adapt macro packages and \CONTEXT\ users are willing to update so it's
not really an issue.

Putting a stamp of approval on a \PDF\ cannot hide the inconsistencies between
for instance vector graphics produced by a third party. They also don't expose
inconsistent use of color and fonts. The page streams produced by \LUATEX\ are
simple and clean enough to not give problems with validation. The problem lays
more with resources coming from elsewhere. When you're phoned by a printing house
about an issue with \RGB\ images in a file where there is no sign of \RGB\ being
used but where a validator reports an issue, you're lucky when an experienced
printer dating back decades then replies that he already had that impression and
will contact the origin. There is no easy way out of this but educating users
(authors) is an option. However, they are often dependent on the publishers and
departments that deal with these and those tend to come with directives that the
authors cannot really argue with (or about).

\stopsection

\startsection[title={Interactivity}]

This is an area where \TEX\ (an therefore also \CONTEXT) always had an edge,
There is a lot possible and in principle all that \PDF\ provides can be
supported. But the more fancy one goes, the more one depends on Acrobat.
Interactivity in \PDF\ evolved stepwise and is mostly market driven. As a result
it is (or was) not always consistent. This is partly due to the fact that we have
a chicken|-|egg issue: you need typesetting machinery, viewer as well as a
standard.

The regular hyperlinks, page or named driven are normally supported by viewers.
Some redefined named destinations (like going to a next page, or going back in a
chain of followed links) not always. Launching applications, as it also relates
to security, can be qualified as an unreliable mechanism. More advanced linking,
for instance using \JAVASCRIPT\ is hardly supported. In that respect \PDF\
viewers lag way behind \HTML\ browsers. I understand that there can be security
risks involved. It's interesting to see that in Acrobat one can mess with
internals of files which makes the \API\ large and complex, but if we stick to
the useful core, the amount of interfacing needed is quite small. Lack of support
in open source viewers (we're talking of about two decades now) made me loose
interest in these features but they are and will be supported in \CONTEXT. We'll
see if and when viewers catch up.

Comments and attachments are also part of interactivity and of course we
supported them right from the start. Some free viewers also support them by now.
Personally I never use comments but they can be handy for popping up information
or embedding snippets or (structured) sources (like \MATHML\ or bibliographic
data). In \CONTEXT\ we can even support \PDF\ inclusion with (a reasonable)
subset of these so called annotations. As the \PDF\ standard no longer evolves
much we can expect all these features to become stable.

\stopsection

\startsection[title={Summary}]

We have always supported the fancy \PDF\ features and we will continue doing so
in \CONTEXT . However, many of them depends on what viewers support, and after
decades of \PDF\ that is still kind of disappointing, which is not that
motivating. We'll see what happens.

\stopsection

\stopchapter
