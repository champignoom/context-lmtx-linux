% language=us

% todo: callback: stream, llx, lly, urx, ury, wd, lsb
% add glyphs runtime
% create whole cff so that we can go from mp

\startcomponent onandon-variable

\environment onandon-environment

\startchapter[title=Variable fonts]

\startsubject[title=Introduction]

History shows the tendency to recycle ideas. Often, quite some effort is made by
historians to figure out what really happened, not just long ago, when nothing
was written down and we have to do with stories or pictures at most, but also in
recent times. Descriptions can be conflicting, puzzling, incomplete, partially
lost, biased, \unknown

Just as language was invented (or evolved) several times, so were scripts. The
same might be true for rendering scripts on a medium. Semaphores came and went
within decades, and how many people know now that they existed and that encryption
was involved? Are the old printing presses truly the old ones, or are older
examples simply gone? One of the nice aspects of the internet is that one can now
more easily discover similar solutions to the same problem but with a different
(and independent) origin.

So, how about this \quotation {next big thing} in font technology: variable fonts?
In this case, history shows that it's not that new. For most \TEX\ users, the
names \METAFONT\ and \METAPOST\ will ring bells. They have a very well-documented
history, so there is not much left to speculation. There are articles, books,
pictures, examples, sources, and more around for decades. So, the ability to
change the appearance of a glyph in a font depending on some parameters is not
new. What probably {\em is} new is that creating variable fonts is done in the
natural environment, where fonts are designed: an interactive program. The
\METAFONT\ toolkit demands quite some insight into programming shapes in such a way
that one can change look and feel depending on parameters. There are not that
many metafonts made, and one reason is that making them requires a certain mind-
and skillset. On the other hand, faster computers, interactive programs,
evolving web technologies, where real-time rendering and therefore more or less
real-time tweaking of fonts is a realistic option, all play a role in acceptance.

But do interactive font design programs make this easier? You still need to
translate ideas into usable beautiful fonts. Taking the common shapes of glyphs,
defining extremes and letting a program calculate some interpolations will not
always give good results. It's like morphing a picture of your baby's face into
yours of old age or that of your grandparent: not all intermediate results will
look great. It's good to notice that variable fonts are a revival of existing
techniques and ideas used in, for instance, multiple master fonts. The details
might matter even more as they can now be exaggerated when transformations are
applied.

There is currently (March 2017) not much information about these fonts, so what I
say next may be partially wrong or at least different from what is intended. The
perspective will be one of a \TEX\ user and coder. Whatever you think of them,
these fonts will be out there, and, for sure, there will be nice examples
circulating soon. And so, when I ran into a few experimental fonts with
\POSTSCRIPT\ and \TRUETYPE\ outlines, I decided to have a look at what is inside.
After all, because it's visual, it's also fun to play with. Let's stress that, at
the moment of this writing, I only have a few simple fonts available, fonts that
are designed for testing and not for usage. Some recommended tables were missing
and no complex \OPENTYPE\ features were used in these fonts.

\stopsubject

\startsubject[title=The specification]

I'm not that good at reading specifications, first of all because I quickly fall
asleep with such documents, but mostly because I prefer reading other stuff
(I do have lots of books waiting to be read). I'm also someone who has to play
with something in order to understand it: trial and error is my modus operandi.
Eventually, it's my intended usage that drives the interface and that is when
everything comes together.

Exploring this technology comes down to: locate a font, get the \OPENTYPE\ 1.8
specification from the \MICROSOFT\ website, and try to figure out what is in the
font. When I had a rough idea, the next step was to get to the shapes and see if
I could manipulate them. Of course, it helped that we can already load fonts and
play with shapes in \CONTEXT using \METAPOST. I didn't have to install and learn
other programs. Once I could render them, in this case by creating a virtual font
with inline \PDF\ literals, a next step was to apply variation. Then came the
first experiments with a possible user interface. Seeing more variation then
drove the exploration of additional properties needed for typesetting, like
features.

The main extension to the data packaged in a font file concerns the (to be
discussed) axis along which variable fonts operate and deltas to be applied to
coordinates. The \type {gdef} table has been extended and contains information
that is used in \type {gpos} features. There are new \type {hvar}, \type {vvar}
and \type {mvar} tables that influence the horizontal, vertical, and general font
dimensions. The \type {gvar} table is used for \TRUETYPE\ variants, while the
\type {cff2} table replaces the \type {cff} table for \OPENTYPE\ \POSTSCRIPT\
outlines. The \type {avar} and \type {stat} tables contain some
metainformation about the axes of variations.

It must be said that, because this is a new technology, the information in the
standard is not always easy to understand. The fact that we have two rendering
techniques, \POSTSCRIPT\ \type {cff} and \TRUETYPE\ \type {ttf}, also means that
we have different information and perspectives. But this situation is not much
different from \OPENTYPE\ standards a few years ago: it takes time, but, in the
end, I will get there. And, after all, users also complain about the lack of
documentation for \CONTEXT, so who am I to complain? In fact, it will be those
\CONTEXT\ users who will provide feedback and make the implementation better in
the~end.

\stopsubject

\startsubject[title=Loading]

Before we discuss some details, it will be useful to summarize what the font
loader does when a user requests a font at a certain size and with specific
features enabled. When a font is used for the first time, its binary format is
converted into a form that makes it suitable for use in \CONTEXT\ and therefore
in \LUATEX. This conversion involves collecting the properties of the font as a
whole (official names, general dimensions like x-height and em-width, etc.), of
glyphs (dimensions, \UNICODE\ properties, optional math properties), and all
kinds of information that relate to (contextual) replacements of glyphs (small
caps, oldstyle, scripts like Arabic) and positioning (kerning, anchoring marks,
etc.). In the \CONTEXT\ font loader, this conversion is done in \LUA.

The result is stored in a condensed format in a cache, and, the next time the font
is needed, it loads in an instant. In the cached version, the dimensions are
untouched, so a font at different sizes has just one copy in the cache. Often, a
font is needed at several sizes, and for each size, we create a copy with scaled
glyph dimensions. The feature-related dimensions (kerning, anchoring, etc.)\ are
shared and scaled when needed. This happens when sequences of characters in the
node list get converted into sequences of glyphs. We could do the same with glyph
dimensions, but one reason for having a scaled copy is that this copy can also
contain virtual glyphs, and these have to be scaled beforehand. In practice, there
are several layers of caching in order to keep the memory footprint within
reasonable bounds. \footnote {In retrospect, one can wonder if that makes sense;
just look at how much memory a browser uses when it has been open for some time.
In the beginning of \LUATEX, users wondered about caching fonts, but again, just
look at what amounts browsers cache: it gets pretty close to the average amount
of writes that a \SSD\ can handle per day within its guarantee.}

When the font is actually used, interaction between characters is resolved using
the feature|-|related information. When, for instance, two characters need to be
kerned, a lookup results in the injection of a kern, scaled from general
dimensions to the current size of the font.

When the outlines of glyphs are needed in \METAFUN, the font is also converted
from its binary form to something in \LUA, but this time we filter the shapes.
For a \type {cff}, this comes down to interpreting the \type {charstrings} and
reducing the complexity to \type {moveto}, \type {lineto}, and \type {curveto}
operators. In the process, subroutines are inlined. The result is something that
\METAPOST\ is happy with but that also can be turned into a piece of a \PDF.

We now come to what a variable font actually is: a basic design which is
transformed along one or more axes. A simple example is wider shapes:

\startlinecorrection
\startMPcode
for i=1 upto 4 :
    fill fullsquare xyscaled (i*.5cm,1cm) shifted (i*2.5*cm,0) withcolor "darkgray";
    fill fullcircle xyscaled (2.5mm,2.5mm) shifted (i*2.5*cm,0) withcolor "lightgray" ;
endfor ;
\stopMPcode
\stoplinecorrection

We can also go taller and retain the width:

\startlinecorrection
\startMPcode
for i=1 upto 4 :
    fill fullsquare xyscaled (1cm,i*.5cm) shifted (i*2.5*cm,0) withcolor "darkgray";
    fill fullcircle xyscaled (2.5mm,2.5mm) shifted (i*2.5*cm,0) withcolor "lightgray" ;
endfor ;
\stopMPcode
\stoplinecorrection

Here we have linear scaling, but glyphs are not normally done that way. There
are font collections out there with lots of intermediate variants (say from light
to heavy) and it's more profitable to sell each variant independently. However,
there is often some logic behind it, probably supported by programs that
designers use, so why not build that logic into the font and have one file that
represents many intermediate forms. In fact, once we have multiple axes, even
when the designer has clear ideas of the intended usage, nothing will prevent
users from tinkering with the axis properties in ways that will fulfil their
demands (and hurt the designer's eyes). I will not discuss that dilemma here.

When a variable font follows the route described above, we face a problem. When
you load a \TRUETYPE\ font, it will just work. The glyphs are packaged in the
same format as static fonts. However, a variable font has axes, and, on each
axis, a value can be set. Each axis has a minimum, a maximum, and a default. It
can be that the default instance also assumes some transformations are applied.
The standard recommends adding tables to describe these things, but the fonts
that I played with each lacked such tables. So that leaves some guesswork. But
still, just loading a \TRUETYPE\ font gives some sort of outcome, although the
dimensions (widths) might be weird due to the lack of a (default) axis being
applied.

An \OPENTYPE\ font with \POSTSCRIPT\ outlines is different: the internal \type
{cff} format has been upgraded to \type {cff2}, which on the one hand is less
complicated, but on the other hand has a few new operators, which results
in programs that have not been adapted complaining or simply crashing on them.

One could argue that a font is just a resource and that one only has to pass it
along, but that's not what works well in practice. Take \LUATEX. We can, of
course, load the font and apply axis vales, so that we can process the document
as we normally do. But, at some point, we have to create a \PDF\ file. We can
simply embed the \TRUETYPE\ files, but no axis values are applied. This is
because, even if we add the relevant information, there is no way in the current
\PDF\ formats to deal with it. For that, we should be able to pass all relevant
axis|-|related information as well as specify what values to use along these
axes. And, for \TRUETYPE\ fonts, this information is not part of the shape
description so then we need to filter and pass more. An \OPENTYPE\ \POSTSCRIPT\
font is much cleaner, because there we have the information needed to transform
the shape mostly in the glyph description. There, we only need to carry some
extra information on how to apply these so|-|called blend values. The
region|/|axis model used there only demands passing a relatively simple table
(stripped down to what we need). But, as said above, \type {cff2} is not
backward-compatible, so a viewer will (currently) simply not show anything.

Recalling how we load fonts, how does that change with variable changes? If we
have two characters with glyphs that get transformed and that have a kern between
them, the kern may or may not transform. So, when we choose values on an axis,
then not only glyph properties change but also relations. We can no longer share
positional information and scale afterwards, because each instance can have
different values to start with. We could carry all that information around and
apply it at runtime, but, because we're typesetting documents with a static design,
it's more convenient to just apply it once and create an instance. We can use the
same caching as mentioned before, but each chosen instance (provided by the font
or made up by user specifications) is kept in the cache. As a consequence, using
a variable font has no overhead, apart from initial caching.

So, having dealt with that, how do we proceed? Processing a font is not different
from what we already had. However, I would not be surprised if users are not
always satisfied with, for instance, kerning, because in such fonts, a lot of care
has to be given to this by the designer. Of course, I can imagine that programs
used to create fonts deal with this, but even then, there is a visual aspect to
it, too. The good news is that in \CONTEXT\ we can manipulate features, so, in
theory, one can create a so|-|called font goodie file for a specific instance.

\stopsubject

\startsubject[title=Shapes]

For \OPENTYPE\ \POSTSCRIPT\ shapes, we always have to do a dummy rendering in
order to get the right bounding box information. For \TRUETYPE, this information
is already present but not when we use a variable instance, so I had to do a bit
of coding for that. Here we face a problem. For \TEX, we need the width, height
and depth of a glyph. Consider the following case:

\startlinecorrection
\startMPcode
path p ; p := fullcircle xysized (3cm,2cm) ;
fill p
    withcolor "lightgray" ;
draw boundingbox currentpicture
    withpen pencircle scaled .5mm
    withcolor "darkgray" ;
setbounds currentpicture to p ;
draw boundingbox currentpicture
    leftenlarged 2mm rightenlarged 5mm ;
\stopMPcode
\stoplinecorrection

The shape has a bounding box that fits the shape. However, its left corner is not
at the origin. So, when we calculate a tight bounding box, we cannot use it for
actually positioning the glyph. We do use it (for horizontal scripts) to get the
height and depth, but for the width, we depend on an explicit value. In \OPENTYPE\
\POSTSCRIPT, we have the width available, and how the shape is positioned relative
to the origin doesn't much matter. In a \TRUETYPE\ shape, a bounding box is part
of the specification, as is the width, but for a variable font, one has to use
so|-|called phantom points to recalculate the width, and the test fonts I had were
not suitable for investigating this.

At any rate, once I could generate documents with typeset text using variable
fonts, it was time to start thinking about a user interface. A variable font
can have predefined instances, but, of course, a user also wants to mess with axis
values. Take one of the test fonts: Adobe Variable Font Prototype. It has several
instances:

\unexpanded\def\SampleFont#1#2#3%
 {\NC #2
  \NC \definedfont[name:#1#3*default]It looks like this!
      \normalexpanded{\noexpand\NC\currentfontinstancespec}
  \NC \NR}

\starttabulate[|||T|]
\SampleFont {adobevariablefontprototype} {extralight}            {extralight}
\SampleFont {adobevariablefontprototype} {light}                 {light}
\SampleFont {adobevariablefontprototype} {regular}               {regular}
\SampleFont {adobevariablefontprototype} {semibold}              {semibold}
\SampleFont {adobevariablefontprototype} {bold}                  {bold}
\SampleFont {adobevariablefontprototype} {black high contrast}   {blackhighcontrast}
\SampleFont {adobevariablefontprototype} {black medium contrast} {blackmediumcontrast}
\SampleFont {adobevariablefontprototype} {black}                 {black}
\stoptabulate

Such an instance is accessed with:

\starttyping
\definefont
  [MyLightFont]
  [name:adobevariablefontprototypelight*default]
\stoptyping

The Avenir Next variable demo font (currently) provides:

\starttabulate[|||T|]
\SampleFont {avenirnextvariable} {regular}           {regular}
\SampleFont {avenirnextvariable} {medium}            {medium}
\SampleFont {avenirnextvariable} {bold}              {bold}
\SampleFont {avenirnextvariable} {heavy}             {heavy}
\SampleFont {avenirnextvariable} {condensed}         {condensed}
\SampleFont {avenirnextvariable} {medium condensed}  {mediumcondensed}
\SampleFont {avenirnextvariable} {bold condensed}    {boldcondensed}
\SampleFont {avenirnextvariable} {heavy condensed}   {heavycondensed}
\stoptabulate

Before we continue, I will show a few examples of variable shapes. Here, we use some
\METAFUN\ magic. Just take these definitions for granted.

\startbuffer[a]
\startMPcode
  draw outlinetext.b
    ("\definedfont[name:adobevariablefontprototypeextralight]foo@bar")
    (withcolor "gray")
    (withcolor red withpen pencircle scaled 1/10)
    xsized .45TextWidth ;
\stopMPcode
\stopbuffer

\startbuffer[b]
\startMPcode
  draw outlinetext.b
    ("\definedfont[name:adobevariablefontprototypelight]foo@bar")
    (withcolor "gray")
    (withcolor red withpen pencircle scaled 1/10)
    xsized .45TextWidth ;
\stopMPcode
\stopbuffer

\startbuffer[c]
\startMPcode
  draw outlinetext.b
    ("\definedfont[name:adobevariablefontprototypebold]foo@bar")
    (withcolor "gray")
    (withcolor red withpen pencircle scaled 1/10)
    xsized .45TextWidth ;
\stopMPcode
\stopbuffer

\startbuffer[d]
\startMPcode
  draw outlinetext.b
    ("\definefontfeature[whatever][axis={weight:350}]%
      \definedfont[name:adobevariablefontprototype*whatever]foo@bar")
    (withcolor "gray")
    (withcolor red withpen pencircle scaled 1/10)
    xsized .45TextWidth ;
\stopMPcode
\stopbuffer

\typebuffer[a,b,c,d]

The results are shown in \in {figure} [fig:whatever:1]. What we see here is that
as long as we fill the shape everything will look as expected, but only using an
outline won't. The crucial (control) points are moved to different locations
and, as a result, they can end up inside the shape. Giving up outlines is the price
we evidently need to pay. Of course this is not unique for variable fonts,
although, in practice, static fonts behave better. To some extent, we're back to
where we were with \METAFONT\ and (for instance) Computer Modern: because these
originate in bitmaps (and probably use similar design logic) we also can have
overlap and bits and pieces pasted together and no one will notice that. The
first outline variants of Computer Modern also had such artifacts, while in the
static Latin Modern successors, outlines were cleaned up.

\startplacefigure[title=Four variants,reference=fig:whatever:1]
    \startcombination[2*2]
        {\getbuffer[a]} {a}
        {\getbuffer[b]} {b}
        {\getbuffer[c]} {d}
        {\getbuffer[d]} {c}
    \stopcombination
\stopplacefigure

The fact that we need to preprocess an instance, but that we only know how to do
that after we have retrieved information about axis values from the font means
that the font handler has to be adapted to keep caching correct. Another
definition is:

\starttyping
\definefontfeature
  [lightdefault]
  [default]
  [axis={weight:230,contrast:50}]

\definefont
  [MyLightFont]
  [name:adobevariablefontprototype*lightdefault]
\stoptyping

Here, the complication is that where normally features are dealt with after
loading, the axis feature is part of the preparation (and caching). If you want
the virtual font solution, you can do this:

\starttyping
\definefontfeature
  [inlinelightdefault]
  [default]
  [axis={weight:230,contrast:50},
   variableshapes=yes]

\definefont
  [MyLightFont]
  [name:adobevariablefontprototype*inlinelightdefault]
\stoptyping

When playing with these fonts, it was hard to see if loading was done right. For
instance, not all values make sense. It is beyond the scope of this article, but
axes like weight, width, contrast, and italic values get applied differently to
so|-|called regions (subspaces). So, say that we have an $x$ coordinate with the
value $50$. This value can be adapted in, for instance, four subspaces (regions),
so we actually get:

\startformula
    x^\prime = x
             + s_1 \times x_1
             + s_2 \times x_2
             + s_3 \times x_3
             + s_4 \times x_4
\stopformula

The (here four) scale factors $s_n$ are determined by the axis value. Each axis
has some rules about how to map the values $230$ for weight and $50$ for contrast
to such a factor. Each region has its own translation from axis values to these
factors. The deltas $x_1,\dots,x_4$ are provided by the font. In a
\POSTSCRIPT|-|based font, we find sequences like:

\starttyping
1 <setvstore>
120 [10 -30 40 -60] 1 <blend> ... <operator>
100 120 [10 -30 40 -60] [30 -10 -30 20] 2 <blend> .. <operator>
\stoptyping

A store refers to a region specification. From there the factors are calculated
using the chosen values on the axis. The deltas are part of the glyphs
specification. Officially, there can be multiple region specifications, but how
likely it is that they will be used in real fonts is an open question.

In \TRUETYPE\ fonts, the deltas are not in the glyph specification but in a
dedicated \type {gvar} table.

\starttyping
apply x deltas [10 -30 40 -60] to x 120
apply y deltas [30 -10 -30 20] to y 100
\stoptyping

Here, the deltas come from tables outside the glyph specification and their
application is triggered by a combination of axis values and regions.

The following two examples use Avenir Next Variable and demonstrate that kerning
is adapted to the variant.

\startbuffer
\definefontfeature
  [default:shaped]
  [default]
  [axis={width:10}]

\definefont
  [SomeFont]
  [file:avenirnextvariable*default:shaped]
\stopbuffer

\typebuffer \getbuffer

\start \showglyphs \showfontkerns \SomeFont \input zapf \wordright{Hermann Zapf}\par \stop

\startbuffer
\definefontfeature
  [default:shaped]
  [default]
  [axis={width:100}]

\definefont
  [SomeFont]
  [file:avenirnextvariable*default:shaped]
\stopbuffer

\typebuffer \getbuffer

\start \showglyphs \showfontkerns \SomeFont \input zapf \wordright{Hermann Zapf}\par \stop

\stopsubject

\startsubject[title=Embedding]

Once we're done typesetting and a \PDF\ file has to be created, there are three
possible routes:

\startitemize
    \startitem
        We can embed the shapes as \PDF\ images (inline literal) using virtual
        font technology. We cannot use so|-|called xforms here, because we want to
        support color selectively in text.
    \stopitem
    \startitem
        We can wait till the \PDF\ format supports such fonts, which might
        happen, but even then we might be stuck for years with viewers getting
        there. Also, documents need to be printed, and when printer support might
        arrive is another unknown.
    \stopitem
    \startitem
        We can embed a regular font with shapes that match the chosen values on the
        axis. This solution is way more efficient than the first.
    \stopitem
\stopitemize

Once I could interpret the right information in the font, the first route was the
way to go. A side effect of having a converter for both outline types meant that
it was trivial to create a virtual font at runtime. This option will stay in
\CONTEXT\ as a pseudo|-|feature \type {variableshapes}.

When trying to support variable fonts, I tried to limit the impact on the backend
code. Also, processing features and such was not touched. The inclusion of the
right shapes is done via a callback that requests the blob to be injected in the
\type {cff} or \type {glyf} table. When implementing this, I actually found out
that the \LUATEX\ backend also does some juggling of charstrings to inline
subroutines. In retrospect, I could have learned a few tricks faster by looking
at that code, but I never realized that it was there. Looking at the code again,
it strikes me that the whole inclusion could be done with \LUA\ code, and, some
day, I will give that a try.

\stopsubject

\startsubject[title=Conclusion]

When I first heard about variable fonts, I was confident that when they showed
up, they could be supported. Of course, a specimen was needed to prove this. A
first implementation demonstrates that, indeed, it's no big deal to let \CONTEXT\
with \LUATEX\ handle such fonts. Of course, we need to fill in some gaps, which
can be done once we have complete fonts. And then, of course, users will demand
more control. In the meantime, the helper script that deals with identifying
fonts by name has been extended, and the relevant code has been added to the
distribution. At some point, the \CONTEXT\ Garden will provide the \LUATEX\
binary that has the callback.

I end on a warning note. On the one hand, this technology looks promising, but on
the other hand, one can easily get lost. Most such fonts probably operate over a
well|-|defined domain of values, but, even then, one should be aware of complex
interactions with features like positioning or replacements. Not all combinations
can be tested. It's probably best to stick with fonts that have all the relevant
tables and don't depend on properties of a specific rendering technology.

Although support is now present in the core of \CONTEXT, the official release
will happen at the \CONTEXT\ meeting in 2017. By then, I hope to have tested more
fonts. Maybe the interface will also have been extended by then, because, after
all, \TEX\ is about control.

\stopsubject

\stopchapter

\stopcomponent
