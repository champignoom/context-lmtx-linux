\startcomponent ma-cb-en-chemical

\enablemode[**en-us]

\project ma-cb

\startchapter[title=Chemical stuf,reference=chemical]

\index{chemistry}
\index{chemical equitation}

\Command{\tex{}chemical}

Chemical structures may look very impressive.

\startbuffer
\startchemical[scale=small,width=fit,top=3000,bottom=3000]
  \chemical[SIX,SB2356,DB14,Z2346,SR3,RZ3,SR6,-RZ6,+RZ6]
           [C,N,C,C,H,H]
  \chemical[PB:Z1,ONE,Z0,MOV8,Z0,SB24,DB7,Z27,PE][C,C,CH_3,O]
  \chemical[PB:Z5,ONE,Z0,MOV6,Z0,SB24,DB7,Z47,PE][C,C,H_3C,O]
  \chemical[SR24,RZ24][CH_3,H_3C]
  \bottext{Compound A}
\stopchemical
\stopbuffer

\placeformula[-]
\startformula
  \getbuffer
\stopformula

\CONTEXT\ relies on \METAPOST\ to draw these kind of chemical structures.
Although these chemical structures are defined with only two or three commands,
it takes some practice to get the right results. This is how the input looks:

\typebuffer

Chemical reactions can be typeset within a paragraph or as a display formula with
the \type{\inlinechemical} and \type{\startchemicalformula} commands:

\startbuffer
\definefloat
  [chemicalformula]
  [chemicalformulas]

One of the steps in the Hasselt canal water treatment is the removal of
phosphate by means of a chemical reaction with iron:

\placechemicalformula[none][]{}
  {\startchemicalformula
    \chemical{Fe(OH)_3}{iron hydroxide}
    \chemical{PLUS}
    \chemical{H_3PO_4}{phosphoric acid}
    \chemical{GIVES}{\hphantom{whatever}}
    \chemical{FePO_4}{iron phosphate}
    \chemical{PLUS}
    \chemical{H_2O}{water}
  \stopchemicalformula}

The \inlinechemical{FePO_4} is a solid and precipitates in water. It
is filtered and re-used as a furtilizer resource.
\stopbuffer

\getbuffer

This is defined by:

\typebuffer

The use of the chemical commands is described in the \goto{PPCHTeX Manual}[ url(manual:chemic) ]
and the example manual \goto{Chemical Formulas in \CONTEXT} [ url(manual:chemic-ex) ].

\stopchapter

\stopcomponent


